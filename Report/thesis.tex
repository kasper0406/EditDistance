\documentclass[twoside,11pt,openright]{report}

\usepackage[utf8]{inputenc}
\usepackage[american]{babel}
\usepackage{a4}
\usepackage{hyperref}
\usepackage{natbib}
\usepackage{latexsym}
\usepackage{mathtools}
\usepackage{amssymb}
\usepackage{amsmath}
\usepackage{amsthm}
\usepackage{epsfig}
\usepackage[T1]{fontenc}
\usepackage{lmodern}
\usepackage[labeled]{multibib}
\usepackage{color}
\usepackage{datetime}
\usepackage{epstopdf} 
\usepackage{cleveref}
\usepackage{geometry}
\usepackage{tikz}
\usepackage[]{algorithm2e}
\usepackage{graphicx}
\usepackage{microtype}
\usepackage{varwidth}
\usepackage{subcaption}

\SetAlFnt{\small}
\SetAlCapFnt{\large}
\SetAlCapNameFnt{\large}

\graphicspath{ {../Benchmarks/} }

\renewcommand*\ttdefault{txtt}

\newcommand{\todo}[1]{{\color[rgb]{.5,0,0}\textbf{$\blacktriangleright$#1$\blacktriangleleft$}}}
\newcommand{\DIST}{\operatorname{DIST}}
\newcommand{\EditDist}{\operatorname{EditDist}}

\newcommand{\substr}[3]{#1\langle #2, #3 \rangle}
\newcommand{\str}[3]{#1[#2, #3]}
\newcommand*{\circled}[1]{\tikz[baseline=(char.base)]{
                          \node[shape=circle,draw,inner sep=2pt] (char) {#1};}}

\newcommand{\ceil}[1] {\lceil #1 \rceil}
\newcommand{\SLP}[1] {\mathcal{#1}}

\newcommand{\refbook}[2]{\cite[#1]{DBLP:journals/corr/abs-0707-3619}, #2}
\newcommand{\reftiskin}[2]{\cite[#1]{Tiskin:2010:FDM:1873601.1873704}, #2}

\newcites{A,B}{Primary Bibliography,Secondary Bibliography}

\newtheorem{mydef}{Definition}
\newtheorem{problem}{Problem}
\newtheorem{claim}{Claim}
\newtheorem{lemma}{Lemma}
\newtheorem{theorem}{Theorem}

% see http://imf.au.dk/system/latex/bog/

\crefname{algocfline}{algorithm}{algorithms}

\usetikzlibrary{shapes,arrows,positioning}

\begin{document}

\RestyleAlgo{boxruled}
\LinesNumbered

%%%%%%%%%%%%%%%%%%%%%%%%%%%%%%%%%%%%%%%%%%%%%%%%%%%%%%%%%%%%%%%%%%%%%%%

\pagestyle{empty} 
\pagenumbering{roman} 
\vspace*{\fill}\noindent{\rule{\linewidth}{1mm}\\[4ex]
{\Huge\sf Computation of Edit Distance in Compressed Strings}\\[2ex]
{\huge\sf Kasper Nielsen, 20091182}\\[2ex]
\noindent\rule{\linewidth}{1mm}\\[4ex]
\noindent{\Large\sf Master's Thesis, Computer Science\\[1ex] 
\monthname\ \the\year  \\[1ex] Advisor: Christian Nørgaard Storm Pedersen\\[15ex]}\\[\fill]}
\epsfig{file=logo.eps}\clearpage

%%%%%%%%%%%%%%%%%%%%%%%%%%%%%%%%%%%%%%%%%%%%%%%%%%%%%%%%%%%%%%%%%%%%%%%

\pagestyle{plain}
% \chapter*{Abstract}
% \addcontentsline{toc}{chapter}{Abstract}

% \todo{in English\dots}

% \chapter*{Resum\'e}
% \addcontentsline{toc}{chapter}{Resum\'e}

% \todo{in Danish\dots}

% \chapter*{Acknowledgements}
% \addcontentsline{toc}{chapter}{Acknowledgments}

% \todo{\dots}

% \vspace{2ex}
% \begin{flushright}
%   \emph{Kasper Nielsen,}\\
%   \emph{Aarhus, \today.}
% \end{flushright}

\tableofcontents
\clearpage
\pagenumbering{arabic}
\setcounter{secnumdepth}{2}

%%%%%%%%%%%%%%%%%%%%%%%%%%%%%%%%%%%%%%%%%%%%%%%%%%%%%%%%%%%%%%%%%%%%%%%

\chapter{Introduction}
\label{ch:intro}
Edit distance is a measure of how similar or dissimilar two string are. The edit distance measure is used in a wide range of practical problems, for example automatic spelling correction and DNA/Protein alignment. This thesis will examine if compressing the input strings to straight-line programs (SLPs, see \cref{def:slp}) can speed up the edit distance computation.

The main focus of this thesis is based on the algorithm described in \cite{Gawrychowski:2012:FAC:2422024.2422048} which builds on top of the approaches described in \cite{DBLP:journals/corr/abs-1004-1194} and \cite{DBLP:journals/corr/abs-0707-3619}. To this date, this is the asymptotically best known algorithm in the RAM for computation of edit distance in strings encoded by SLPs. Given two input strings $A$ and $B$ of length respectively $k$ and $l$, the algorithm runs in time $O(n N \sqrt{\log{\frac{N}{n}}})$, where $N = k + l$ and $n$ is the sum of sizes of the compressed strings (a formal definition on how $n$ is measured is given in \cref{ch:compressing-strings}). It is a open problem if the running time can be reduced to $O(nN)$. This is a bit further discussed in \cref{sec:conclusion:further-work}.

This thesis has a two-fold objective. First and foremost to implement the algorithm and benchmark it to see how the technique performs in practice. The second objective is to give a coherent description of the algorithm and argue that it is correct. The proof in order to establish the correctness follow the ones presented in \cite{DBLP:journals/corr/abs-0707-3619}, \cite{Gawrychowski:2012:FAC:2422024.2422048} and \cite{Tiskin:2010:FDM:1873601.1873704}. The proofs supplied in this report are in many cases more thorough, detailed and is a bit closer to the actual implementation.

The algorithm for speeding up the edit distance computation using compressed strings, can be divided into two major steps. The first is compressing the input strings, which is described in \cref{ch:compressing-strings} and the second step is the actual edit distance computation which uses the compressed strings as input. The algorithm for the actual computation is described in \cref{ch:algorithm}.

Both of the chapters are finished by presenting benchmarks. The benchmark section of \cref{ch:compressing-strings} describes how well different types of strings compresses and verifies the theoretical running time of the compression algorithm. Especially the compressibility of different types of inputs is of great interest since it bounds the speedup of the computation. The benchmark section of \cref{ch:algorithm} verifies the theoretical running time of the algorithm, compares how well the compression based algorithm performs to a simple edit distance algorithm and lastly examines the time usage by the different steps (see \cref{sec:edit-dist-on-slps-steps}) of the compression based algorithm.

\section{Overview}
This section is intended to give an overview of the implemented algorithms, how the proofs of correctness relates and the main steps of the algorithm.

The main steps of the implemented algorithm based on string compression are the following:
\begin{itemize}
  \item Compress the input strings $A$ and $B$ to straight-line programs (SLPs) (defined in \cref{def:slp}).
  \item Divide the input strings into blocks of a certain size based on their SLP representation.
  \item Compute a structure called a DIST repository. This is a structure that for every pair of blocks in the two SLPs contains whats known as a DIST table. A DIST table is a representation of how the two blocks compares wrt. the computation.
  \item Compute the actual result by utilizing the DIST repository.
\end{itemize}
In order to gain the speedup a succinct representation of the DIST repository is used. This representation encodes a DIST table as a permutation of the identity matrix and theoretically turns out to facilitate both fast construction and usage of the DIST repository.

The algorithm relies on the following list of sub-algorithms which all have been implemented as part of writing this thesis:
\begin{itemize}
  \item The SLP construction algorithm of \cite{Rytter2003211}. Described in \cref{ch:compressing-strings}.
  \item Algorithm from \cite{DBLP:journals/corr/abs-1004-1194} for dividing the input strings into blocks of size $O(x)$ for some $x$, corresponding to productions in the SLP. Described in \cref{sec:algorithm:select-productions}.
  \item Algorithm from \cite{DBLP:journals/corr/abs-1004-1194} for building the DIST repository based on the blocks of the input string. Described in \cref{sec:algorithm:building-DISTs-overview}.
    \begin{itemize}
      \item This algorithm requires the ability to compute the minimum distance product (see \cref{def:minimum-distance-product}) of two simple unit-Monge matrices (see \cref{def:simple} and \cref{def:unit-monge}) given by \cite{Tiskin:2010:FDM:1873601.1873704}. Described in \cref{sec:algorithm:min-mult-two-unit-monge}.
    \end{itemize}
  \item Algorithm for computing the actual result by using the DIST repository. This algorithm is only superficially described in \cite{Gawrychowski:2012:FAC:2422024.2422048} and \cite{DBLP:journals/corr/abs-0707-3619}. \Cref{sec:algorithm:filling-grid-overview} describes the details of this computation. The algorithm relies on:
    \begin{itemize}
      \item An algorithm from \cite{Gawrychowski:2012:FAC:2422024.2422048} for computing the maximum distance product of a DIST table with an arbitrary vector. Described in \cref{sec:algorithm:max-mult-H-table-with-vector}.
      \item The maximum distance product algorithm further relies on an interval union find data-structure from \cite{Itai06lineartime}. Described in \cref{sec:algorithm:interval-union-find}.
    \end{itemize}
\end{itemize}
The algorithm for constructing the SLP representation of the input strings requires a suffix tree. In order to have a succinct and efficient suffix tree, it has been chosen to use a suffix tree implementation in the Succinct Data Structure Library (SDSL) \cite{SDSL} based on \cite{OHL:FIS:GOG:2010}.

In order to understand and prove the correctness of the algorithm a collection of proofs is described. \Cref{fig:proof-overview} gives an overview of how the main definitions, lemmas and theorems of this thesis relates. Also the figure points to the origin in the literature and is color coded by how much detail is added in this thesis compared to the literature.

\setlength{\tabcolsep}{3pt}
\tikzstyle{block} = [draw, rectangle, align=left]
\newcommand{\nodecontent}[3]{%
  \begin{tabular}{l p{2.4cm}}
    \multicolumn{2}{p{3.5cm}}{#1} \\
    \textit{Reference:} & #2 \\
    \textit{Literature:} & #3 \\
  \end{tabular}
}%

\begin{figure}[h!]
  \begin{tikzpicture}[font=\tiny\sffamily,node distance=5cm]
    \node[block](simple-classification){
      \nodecontent{Characterization of simple matrices}
                  {\Cref{claim:simple-matrix-characterization}}
                  {\cite[p. 1288]{Tiskin:2010:FDM:1873601.1873704}}
    };

    \node[block, right of=simple-classification](unit-Monge-classification){
      \nodecontent{Characterization of unit-Monge matrices}
                  {\Cref{lemma:unit-monge-characterization}}
                  {\cite[Lemma 2.11, p. 10]{DBLP:journals/corr/abs-0707-3619}}
    };

    \node[block, below right of=simple-classification](dist-representation){
      \nodecontent{Succinct DIST representation}
                  {\Cref{lemma:H-permutation-representation}}
                  {\cite[Theorem 4.10, p. 49]{DBLP:journals/corr/abs-0707-3619}}
    };

    \node[block](simple-unit-monge-closed){
      \nodecontent{Simple unit-Monge matrices closed\newline under the minimum distance product}
                  {\Cref{claim:unit-monge-min-prod-closed}}
                  {\cite[Theorem 3.4, p. 16]{Tiskin:2010:FDM:1873601.1873704}}
    };
    \node[block, below of=simple-unit-monge-closed](min-mult){
      \nodecontent{Minimum Distance Product}
                  {\Cref{sec:algorithm:min-mult-two-unit-monge}}
                  {\cite[Theorem 3.3]{Tiskin:2010:FDM:1873601.1873704} and\newline\cite[Theorem 3.16, p. 28]{DBLP:journals/corr/abs-0707-3619}}
    };

    \draw [->] (simple-unit-monge-closed) -- (min-mult);
  \end{tikzpicture}

  \caption{Overview of the proofs needed to describe and prove the correctness of the algorithm. Specifies how the proves in this thesis relates to the ones in the literature.}
  \label{fig:proof-overview}
\end{figure}

\section{Notation and definition of edit-distance}
A string $\str{A}{0}{m}$ denotes a sequence of $m$ characters with random access, where the first is inclusive and the last is exclusive. That is, for $i \in \{0, \dots, m - 1\}$ the $i$'th character in the string can be accessed by $A[i] \in \Sigma$. Throughout this report it is assumed that the alphabet $\Sigma$ is finite.

The substring of a string $\str{A}{0}{m}$ is denoted by $\substr{A}{i}{j} = A[i]A[i+1]\dots A[j - 1]$ where $i, j \in \{0, \dots, m\}$. If $j \leq i$ then $\substr{A}{i}{j}$ denotes the empty string.

\todo{Introduce matrix notation.}

The edit distance problem can now be defined more formally:
\begin{problem}
  \label{problem:edit-distance}
  The edit distance between two strings $\str{A}{0}{m}$ and $\str{B}{0}{n}$ counts the minimum number of operations needed to transform $A$ into $B$. The set of the available operations are to: insert a character, delete a character or to substitute a character by another. Each of the operations is associated with a cost. Throughout the thesis the Levenshtein distance will be used, where the cost of a substitution is 0 on a match and 1 otherwise. Insert and delete operations have cost 1.
\end{problem}
The scores could be generalized to other constants, but also to arbitrary functions. Different choices may result in more complicated algorithms with worse asymptotic running time and space consumption, even if the cost function is assumed to be computable in constant time. However, this thesis will only focus on the edit distance problem as defined in \cref{problem:edit-distance}, and how fast this can be computed in compressed strings.

\section{Test data}
\label{sec:intro:test-data}
In order to test the algorithms, some test data needs to be selected. As the algorithm relies on compression of the input, it is natural to choose test input with different compressibility characteristics. It is also worth noticing that the alphabet size does not directly influence the running time of the algorithm, however a smaller alphabet will result in less compressibility of the strings.

The following inputs have been chosen for benchmarking:
\begin{description}
  \item[Fibonacci strings] A Fibonacci string is a string over the alphabet $\Sigma = \{a,b\}$. The $n$'th Fibonacci string $F_n$ is given by the following recurrence:
    \begin{align*}
      F_1 &= \text{a} \\
      F_2 &= \text{ab} \\
      F_{n} &= F_{n - 2} F_{n - 1} \quad\quad \text{for } n \geq 3
    \end{align*}
    It follows from the definition that these strings are very repetitive, and is therefore expected to compress very well. In \cref{sec:compressing-Fibonacci-strings} it is argued that there exists a SLP of size $O(\log{n})$ generating $F_n$.
  \item[Human genome] The human genome is from the perspective of this thesis a very long DNA sequence over the alphabet $\Sigma = \{A,G,C,T\}$. For practical purposes the human genome, or similar DNA sequences, may be the most realistic type of input. Furthermore DNA are said to contain a lot of repetitions which may result in a smaller SLP generating the string.

   The used human genome is \textbf{hg38} from \cite{HumanGenome}. The genome is divided into two categories: Repeating areas and non-repeating areas. The benchmarks in this report will use the following configurations for testing
  \begin{itemize}
    \item Repeating areas masked out from the genome (\texttt{hg\_repeating})
    \item Non-repeating areas masked out from the genome (\texttt{hg\_nonrepeating})
    \item The entire genome (\texttt{hg\_entire})
  \end{itemize}
  It is expected that this type of input will be less compressible than Fibonacci strings, but better compressible than random strings due to the repetitive structure of DNA.

  \item[Uniformly random strings] This type of strings are generated by sampling every character in the string uniformly random from the alphabet $\Sigma = \{A,G,C,T\}$. This means that all characters are independent from each other, and should therefore be very incompressible in nature. Therefore this type of input should represent worst-case input for the algorithm, since it heavily relies on compression to obtain a speedup.
\end{description}
In \cref{sec:compression:benchmarks} the actual compressibility of all the above input types are investigated. \Cref{sec:algorithm:benchmarks} test how fast the compression based algorithm is able to handle the different input types.

\section{Implementation and test setup}
All measurements presented in this report were performed on a computer with an Intel i5-4258U CPU with the following technical specifications:
\begin{itemize}
  \item 2 cores operating at 2.4GHz, with turbo boost up to 2.9GHz.
  \item 256KB L2 cache, 3MB L3 cache.
\end{itemize}
The main memory size of the machine is 16 Gb. The computer was running Mac OS X and Intel Performance Counter Monitor (IPCM) was used to measure cache misses and instructions executed. All implementations have been written in \texttt{C++11} and compiled on  Mac OS X using the clang 3.3 compiler. When measurements were performed, the code was compiled using the \texttt{-O3 -funroll-loops -flto -fasm-blocks -march=core-avx2 -DNDEBUG} optimization flags.

Before doing the actual measurements, a small warm-up round was performed, ensuring that the code was in the CPU cache etc. This helped to greatly reduce the variance of the running time measurements yielding more consistent results.

All running times was measured as wall-time, and all tests was repeated 5 times where the measurement with the median running times is the one used for data-analysis in this report. The reason for choosing the median measurement, is that an unexpected amount of machine activity in one of the trials should not penalize the result, but at the same time, some tasks done by the operating system may be needed for the algorithm to function, which is why the minimum time was not selected. However, for larger inputs the relative standard deviation was less than $1.5\%$ \todo{Check with final measurements}, meaning that the measurements in each trails are close to each other.

All the tested input sizes have been grown by a factor of $1.7$. This has been chosen to have a exponential growth with a reasonable step size that is not a power of $2$. The reason for not choosing a power of $2$ is that this could give worse measurement results than would be seen for other input sizes, primarily due to cache associativity. Since inputs in practice are typically not strict powers of $2$ the artifacts by cache associativity is not desired.

\subsubsection{Correctness}
In order to verify the correctness of the implementation, the result has been checked up against the simple algorithm described in \cref{sec:intro:simple}. Also, unit tests have been made to individually test parts of the algorithm, and during debugging and testing a lot of consistency checks are performed on every run. There are no known bugs in the implementation.

\chapter{Edit Distance approaches}
There are multiple ways to compute the edit distance between two strings. Some of the relevant contributions with respect to this thesis is mentioned in this chapter.

\section{Simple algorithm}
\label{sec:intro:simple}
A standard dynamic programming algorithm for solving the edit distance problem, here denoted the simple algorithm, was given by Wagner and Fisher. It is given two string $\str{A}{0}{m}$ and $\str{B}{0}{n}$ for which the edit distance should be computed, and fills out a matrix of size $m \times n$ where a given entry $(i, j)$ corresponds to the edit distance between $\str{A}{0}{i}$ and $\str{B}{0}{j}$. The rules for filling out a table entry follows directly from the definition of edit distance when extending a string, and are as follows:
\[
  T[i, j] = \max \left\{ \begin{array}{lll}
              T[i - 1, j - 1] + \delta(A[i - 1], A[j - 1]) & \text{if } i, j \geq 1 & \text{Match} \\
              T[i - 1, j] + 1       & \text{if } i \geq 1, j \geq 0       & \text{Insert} \\
              T[i, j - 1] + 1       & \text{if } i \geq 0, j \geq 1       & \text{Delete}
            \end{array} \right.
\]
where
\[
  \delta(a, b) = \begin{cases}
                   0 & \text{if } a = b \\
                   1 & \text{o/w.}
                 \end{cases}
    \quad \text{ for } \quad a, b \in \Sigma.
\]
The base case where a string is aligned to the empty string is trivially the length of the non-empty string. Therefore the edit distance between two string can be computed using both $O(nm)$ time and space. Notice that an easy optimization where only the last row used is stored in memory reduces the space consumption to $O(\min\{n, m\})$.

\subsection{Backtracking}
If the actual alignment of the strings is desirable, it can be found by a standard back-tracking approach on the computed matrix.

This however implies that the simple space optimization from before does not work, as all entries in the computed table potentially is needed for the backtracking. However, a divide-and-conquer algorithm was given by Hirshberg\todo{Cite} that reduces the space consumption of $O(\min\{n, m\})$ without asymptotically increasing the running time.\todo{Describe the approach?}

The same approach can be used to find the alignment for the main algorithm in this report. However, a more practical approach may be to use the simple algorithm with pruning when the known optimal score is known, and then use backtracking directly on the grid. From now on, only the actual edit distance is considered to be of interest.

\subsection{Optimizations}
Ideas:
\begin{itemize}
  \item Compute matrix only around its diagonal, and then extend if needed. Extend by powers of $2$, which will result in at most a factor of $2$ extra running time.
  \item Using SIMD instructions.
  \item Parallelization (CPU / GPU).
\end{itemize}

In order to bound the scope of this thesis, it has been chosen not to focus on parallelization of the algorithms, even through both the simple and the main algorithm investigated in this thesis of \cite{Gawrychowski:2012:FAC:2422024.2422048} should be easily parallelizable. For more details see \ref{sec:conclusion:further-work}.

\section{Compression based algorithms}
There exists other approaches for speeding up the computation of the edit distance between two strings. This section presents a short overview of some of the approaches that relates to the approach investigated in this thesis.

One approach for speeding up the computation of edit distance, is to compress the input strings, and then in some way do the edit distance computation using the compressed strings, which for many cases will be shorter than the original input.

\subsection{The Four Russian Algorithm}
This approach was not originally presented as a compression based algorithm, but using that each entry in the dynamic programming matrix is changing at most 1 from an adjacent entry, and that the alphabet is finite, it is possible to encode blocks of the dynamic programming table efficiently and pre-compute all of them. That is, for all different sub-strings and possible inputs to compute the relative outputs when applying the block. Due to that values can change by no more than $1$ between two entries, the representation can be succinctly represented and efficiently looked up. The pre-computed tables are then used to fill out the original dynamic programming matrix block-by-block.
\todo{Describe details and analyze time complexity?.}

\subsection{Run-Length Encoding}
\begin{itemize}
  \item \todo{Describe RLE}
  \item Not the focus / implemented in this thesis, but the result is included for completeness. Refer old thesis for experimental results of this approach?
  \item \todo{cite} gave an algorithm running in time $O(nm)$ where $n$ and $m$ denotes the size of the compressed sequences. This is the first result where the running time is only depending on the length of the compressed strings. However, for string not containing long runs of the same symbol, RLE will generate strings longer than the original input!
\end{itemize}

\subsection{Compression based on straight-line programs}
\label{sec:edit-dist-on-slps-steps}
This section gives a very brief overview of the algorithm given by \cite{Gawrychowski:2012:FAC:2422024.2422048}, which is the main focus of this thesis. A more detailed description of each of the steps of the algorithm will be given later in this report. 

Another way of compressing a string, is to compress it to what is known as a straight-line program (SLP). A more formal definition of a SLP will be given in \cref{ch:compressing-strings}, for now it is sufficient to note that a SLP is a restricted context-free grammar.

\begin{figure}
  \centering
  \includegraphics[width=7cm]{images/grid-dist-2}
  \caption{Stacking of $\DIST$-tables showing how they are used to skip the computation of some of the entries in the dynamical programming grid.}
  \label{fig:intro:dist-tables}
\end{figure}

The algorithm can be seen as a generalization of the Four Russian approach, where the blocking of the dynamic programming table is based on the productions of the SLPs (see \cref{fig:intro:dist-tables}), so that reused productions result in reusable blocks when filling out the grid. The algorithm can be summarized in the following steps:
\begin{itemize}
  \item Construct SLP representations $\SLP{A}$, $\SLP{B}$ of the two input strings $\str{A}{0}{m}$ and $\str{B}{0}{n}$. The better the compression scheme is able to compress the SLP representations the faster the algorithm will be. A natural trade-off arises in this step, namely the running time of the compression scheme versus the quality of the resulting compression. This step will be discussed in \cref{ch:compressing-strings}.
  \item Productions from the SLPs generating strings of length $O(x)$ for some fixed $x$ are selected in such a way that they cover the entire string without duplication. This can be done in $O(N)$ time by traversing the productions of the SLP bottom-up as described in \cref{sec:algorithm:select-productions}.
  \item Blocks from the dynamic programming table are precomputed for all the selected productions (these precomputed tables are called DIST tables). By naively representing the DIST tables they each take up $O(x^2)$ space, but using a succinct representation given in \cite{DBLP:journals/corr/abs-0707-3619}, it is possibly to reduce the space consumption to $O(x)$ and also improve the time it takes to use a $\DIST$. This step is described in \cref{sec:algorithm:building-DISTs-overview}.
  \item The dynamical programming table is filled up by using the partition and the precomputed $\DIST$ tables. This step takes time $O\left(\left(\frac{N}{x}\right)^2 \cdot AP(x)\right)$ where $AP(x)$ is the time to apply a DIST table. Using the succinct representation, \cite{Gawrychowski:2012:FAC:2422024.2422048} shows how to make $AP(x) = O(x)$, leading to a total running time of $O(\frac{N^2}{x})$. This step is described in \cref{sec:algorithm:filling-grid-overview}.
\end{itemize}

\chapter{Compressing input strings}
\label{ch:compressing-strings}
There exist a large number of compression schemes for compressing strings, spanning both lossy and lossless schemes. Since it is desired that the edit distance algorithm always gives the correct result, only lossless compression schemes are considered.

The algorithm examined in this thesis is working over straight-line programs (SLPs), whereby only compression to this encoding is considered.
\begin{mydef}
  \label{def:slp}
  A straight-line program (SLP) is a restricted context-free grammar $G = (V, T, R, S)$, where
  \begin{itemize}
    \item $V = \{ S_1, \dots, S_n\}$ is a ordered set of non-terminals
    \item $T = \{ S_{n + 1}, \dots, S_{n + m} \}$ is a ordered set of terminals
    \item $R$ is a set of production rules mapping $V$ to $(V \cup T)^2$, with the condition that if $(S_i \to S_jS_k) \in R$, then $j,k \geq i$. This ensures that a SLP generates exactly one string.
    \item The start production $S = S_1$.
  \end{itemize}
  The size of a SLP is defined to be $|V \cup T|$, which is only a constant from the actual space consumption of a simple pointer-based SLP implementation.
\end{mydef}
From this definition it is not possible for a SLP to generate the empty string, however this poses no problem in practice as that case can be handled separately.
\begin{problem}
  \label{compression:problem:minimum-slp}
  Given an input string $\str{A}{0}{m}$ find a SLP $\SLP{A}$ of minimum size generating $A$.
\end{problem}
It is known \cite[p. 212]{Rytter2003211} that this problem is NP-complete, hence a general, exact and efficient algorithm should not be expected to be easily obtainable. However, a number of articles (ex. \cite{Rytter2003211} and \cite{Sakamoto2005416}) have studied how to approximate a minimum size SLP for the input string. The approach implemented in this article is a $O(\log{n})$-approximation given in \cite{Rytter2003211}, which is based on the Lempel-Ziv factorization (LZ-factorization), which is a version of the popular LZ77 compression scheme. The algorithm is described in \cref{sec:compression:alg}.

A small theoretical improvement was given in \cite{Rytter2003211}, reducing the approximation factor to $O(\log{\frac{n}{g^*}})$ where $g^*$ denotes the size of the minimal SLP compression. As suggested by \cite{Rytter2003211} the improvement is mostly cosmetic, and from the description of the approach, I find it very likely that the approach will significantly increase the constant in front of the big-$O$ notation. Therefore, I have not chosen to implemented this theoretical improvement.

For computing the LZ-factorization a suffix tree is used. Suffix tress often comes with a large constant space overhead, making them undesirable in practice in some cases. \cite{Sakamoto2005416} gave another algorithm for solving \cref{compression:problem:minimum-slp} which also actives the $O(\log{\frac{n}{g^*}})$ approximation factor, but does it based on the RE-PAIR encoding scheme, which can be made to work without using a suffix tree. The reason for not implementing this approach, is that I find the LZ-factorization approach more intuitive, and for the benchmarks in this report the space consumption of the suffix tree has not been a problem.

In other areas of lossless compression, ex. bzip2 compression, the input string is encoded using a Burrows-Wheeler transform (BWT) \cite{BurrowsWheeler}. The intuition why the BWT works, is that it permutes the string in such a way that highly repetitive parts of the string are group together, making them easier to compress. However building the SLP over the BWT of the input string will not work because the BTW permutes the string, such that productions of the SLP does not corresponds to a continuous area in the input string. I have not been able to find any articles able to take advantage of the BWT for SLP based compression.

In lossless compression there exist Entropy-based encoding schemes. Such schemes uses the frequencies of the appearing alphabet symbols in the string to compress them to use as few bits as possible. This type of compression is not of significant interest in this context, since we are only interest in minimizing the number of productions in the resulting SLP.

\paragraph{Notes about size}
Given input strings over the finite alphabet $\Sigma$, all strings over this alphabet can trivially be encoded using $\ceil{\log_2 |\Sigma|}$ bits per symbol. However, for a direct representation of the SLP using pointers a left and a right pointer has to be stored together with the symbol, which on a new machine uses up $64$ bits each. Moreover, the algorithms working on the SLPs are likely to require more information annotated to each node in the tree. This results in a huge constant blow up on the space required to store the input string, especially for strings not compressing very well.

It could be both of practical and theoretical interest to try to find a succinct representation of a SLP, in order to make the space consumption smaller, making the SLP approach practical for long strings that does not compress very well.

\section{Algorithm}
\label{sec:compression:alg}
The idea behind the algorithm is to iteratively build the SLP for the string. This incremental building process will be based on repeating parts of the string called the LZ-factorization, which will be described after the overview of the algorithm has been given. The overall algorithm is given in \cref{alg:compression:approx}.

\begin{algorithm}[H]
  \label{alg:compression:approx}
  Let $\SLP{A}$ be an empty SLP\;
  Construct LZ-Factorization $f_1,\dots,f_k$ of the input string $\str{A}{0}{m}$\;
  \For{$i = 1, \dots, k$}{
    Find productions $S_1,\dots,S_{k_i}$ in the current SLP $\SLP{A}$ generating $f_i$\;
    Set $\SLP{S} = Concat(S_1,\dots,S_{k_i})$\;
    $\SLP{A} = Concat(\SLP{A}, \SLP{S})$\;
  }
  \Return $\SLP{A}$
  \caption{Algorithm for generating $\log{n}$ approximation of minimal SLP of input string $\str{A}{0}{m}$.}
\end{algorithm}

When building the SLP, the SLP will be considered as an AVL-tree. Therefore it is known how to search and concatenate two SLPs in a way that keeps the resulting SLP balanced. The idea is to add add a repeating factor in the string once at a time. The repeating factor is added by reusing the productions already in the SLP generating the factor. During the concatenating operation of SLPs, the resulting tree is balanced using appropriate AVL-tree rotations.

The $Concat$ process referenced in the code, is function almost the same way as concatenating two AVL trees \todo{ref?}. However, productions in the SLP may be shared, hence when adding nodes and making rotations, special care needs to be taken in order to ensure the balance of the tree and that the derived string is correct. This is easy to fix by simple introducing a constant number of new productions for every insertion and rotation. Amortized, the number of rotations that happens on every concatenation is constant and therefore only a constant number of extra productions are added every time $Concat$ is called.

The factors iteratively added to the SLP are the LZ-factors of the input string $\str{A}{0}{m}$. The LZ-factors of a string is found by scanning the string from left to right. Assume that the current position is $i \in \{0, \dots, m - 1\}$, then the next LZ factor is found by finding the longest string starting at position $i$ that have occurred in the input string before position $i$. \Cref{fig:lz-factor} illustrates the situation.

\begin{figure}[htb]
  \centering
  \includegraphics[width=8cm]{images/lz-factor}
  \caption{A graphical illustration of a LZ-factor.}
  \label{fig:lz-factor}
\end{figure}

It is required to be able to efficiently find the nodes in the current SLP which generates a LZ factor. In order to do this, the LZ factors of the string are specified by their first occurrence in the input string by indices. Using these indices it is easy to find the corresponding productions in the SLP by annotating the derived length for each of the productions, and then make a top-down decent of the tree checking if the indices overlaps the current node.

The only step of the SLP construction algorithm not thoroughly described in \cite{Rytter2003211}, is how to compute the LZ-factorization of a string efficiently. They only state that it can be done efficiently using a suffix tree. Although the algorithm is quite simple, it is included in this report for completeness.

First the suffix tree for the input string is generated. Then every node $v$ in the tree is annotated with the first index in the string where the substring spelled out from the root to $v$ occurs. This is done by traversing the tree bottom up where all leafs are marked by their suffix numbers and for any parent the minimum of their children is chosen.

Having the described suffix tree, the LZ-factorization can be found by scanning the input string and follow the corresponding path in the suffix tree. While the string has already occurred before the path is extended, and when it is not a new LZ factor is added and the search begins from the top of the suffix tree. Pseudo code for the algorithm is given in \cref{alg:compression:lz-factorization}. The algorithm could potentially be improved a bit by following an entire suffix tree edge in every iteration. However, since the SLP construction is not the time consuming part of the algorithm this has not been implemented as the gain will be almost non noticeable.

\begin{algorithm}[htb]
  \label{alg:compression:lz-factorization}
  $LZFactors = \emptyset$\;
  $factorStart = 0$\;
  \While{$factorStart < |A|$}{
    $previousNode = currentNode = Root$\;
    $currentEdgeLength = currentEdgeMatched = 0$\;

    $factorLen = 1$\;
    \While{true}{
      \If{$currentEdgeMatched \geq currentEdgeLength$}{
        $previousNode = currentNode$\;
        $pos = factorStart + factorLen - 1$\;
        $currentNode, followedEdge = \operatorname{FastScan}(currentNode, A[pos])$\;
        $currentEdgeLength = |followedEdge|$\;
        $currentEdgeMatched = 0$\;
      }

      \If{$FirstSeen[Root \to currentNode] + factorLen - 1 \geq factorStart$}{
        \eIf{$factorLen == 1$}{
          $LZFactors = LZFactors \cup \{ (factorStart, factorStart) \}$\;
          $factorStart = factorStart + 1$\;
        }{
          $pos = FirstSeen[Root \to previousNode]$\;
          $LZFactors = LZFators \cup \{ (pos, pos + factorLen - 2) \}$\;
          $factorStart = factorStart + factorLen - 1$\;
        }

        break\;
      }

      $previousNode = currentNode$\;
      $currentEdgeMatched = currentEdgeMatched + 1$\;
      $factorLen = factorLen + 1$\;
    }
  }
  \Return $LZFactors$
  \caption{Algorithm for generating the LZ-factorization of a given input string $A$. The algorithm assumes a $FirstSeen$ array that for a given node gives the first time its associated string occurs in the string.}
\end{algorithm}

The implementation uses a pointer based representation of the SLP. It is worth noticing that in the context of standard binary heaps pointer based implementation typically performs worse than an implicit array representation due to the extra overhead of lookup pointers and jumping in memory. In a previous project \cite[p. 9]{AA13Project1} the penalty of using pointers was found to be around $2$.

It is also expected that a pointer based SLP will suffer for such a penalty when traversing the tree. However, since the SLP does not have a fixed topology, the pointer based solution is the easiest way to represent the topology. It may be the case that this could be encoded in some clever way in combination with a succinct representation of a SLP, but this is outside the scope of this thesis.

\clearpage
\section{Benchmarks}
\label{sec:compression:benchmarks}
This section will examine how well different types of string can be compressed to a SLP. It will also check that the theoretical running time of $O(n\log{n})$ is also satisfied in practice.

\subsection{Compression quality}
First the quality of the compressions is examined for the different type of inputs as described in \cref{sec:intro:test-data}.

\subsubsection{Fibonacci strings}
\label{sec:compressing-Fibonacci-strings}
When evaluating the quality of compressed Fibonacci strings, it is useful to consider how well they can be compressed to a SLP. For this reason recall the definition of Fibonacci strings:
\begin{align*}
  F_1 &= \text{a} \\
  F_2 &= \text{ab} \\
  F_{i} &= F_{i - 2} F_{i - 1} \quad\quad \text{for } i \geq 3
\end{align*}
Notice directly by definition, $F_1$ can be represented by a SLP of size $1$ and $F_2$ can be represented by a SLP of size $3$. Constructing $F_{i + 1}$ for $i > 2$ can be done by adding one extra production to the SLP describing $F_{i}$. Notice that no smaller SLP can describe the Fibonacci strings, since $F_{i+1}$ is a super string of $F_i$, and by induction the SLP representation of $F_i$ is optimal.

Therefore Fibonacci strings compresses very well to a SLP, where a representation of $F_{i}$ uses only $i + 1$ productions for $i \geq 2$. Therefore the quality of the compressed Fibonacci strings can be measured relative to the optimal SLP. \Cref{fig:compression:quality:fibonacci} shows a plot where the blue line shows how big a factor the computed SLP is bigger than the optimal SLP.

By the proof in \cite{Rytter2003211} this overhead should be bounded by $O(\log{n})$. The red line of \cref{fig:compression:quality:fibonacci} plots the production overhead normalized by $\log{n}$. From theory this should converge towards a constant, which also seems to be the case in practice.
%
\begin{figure}[h!]
  \centering
  \includegraphics[width=10cm]{compression/fib}
  \caption{Plot of how well Fibonacci strings compresses to a SLP relative to the minimal SLP describing the corresponding Fibonacci string.}
  \label{fig:compression:quality:fibonacci}
\end{figure}

\subsubsection{Random strings}
In these benchmarks the compressibility as function of sequence length has been examined by selecting prefixes of the input string, and then measured how much a prefix of a certain length is able to be compressed. For random input this should be alright, since the starting position does not affect the distribution.

Random strings contains no systematic patterns, and will by design maximize the average information rate \todo{describe/reference?}, which by information theory should make them very hard to compress. The only type of compression in this type of string is expected to come since the alphabet has a finite size and the strings are of arbitrary length.

It can be seen on \cref{fig:compression:quality:random} that the compression ratio of the string is decreasing as the input size gets larger. This is also expected due to the fixed alphabet size, and this is also why the Four Russian approach sketched in \todo{ref} works.
\todo{Comment on slope, and relate to theory?}
%
\begin{figure}[h!]
  \centering
  \includegraphics[width=10cm]{compression/random}
  \caption{Plot of compression quality of random strings.}
  \label{fig:compression:quality:random}
\end{figure}

\subsubsection{DNA sequences}
It has been chosen to adopt the same way of benchmarking compressibility as for the random strings. That is, a prefix of the DNA sequences has been selected and tested for compressibility. The choice of always selecting a prefix may be a bit misleading for small prefix lengths since non-coding regions have a more random distribution than coding regions \todo{cite}. However, as the selected prefixes gets longer this should even out, and the actual compressibility of the string should be accurately measured.

It can be seen on \cref{fig:compression:quality:hg} that only a small improvement in the compressibility of the input strings are gained by using the human genome \cref{sec:intro:test-data} instead of random strings. \todo{Consider using DNA sequences that are more repetitive.}

Since random strings should not compress very well, it is not desirable that the DNA sequences only compresses a little better. However, the compression factor is still around $0.5$ for large input, and therefore there is still hope that the edit distance algorithm may still perform better than the simple.

\begin{figure}[h!]
  \centering
  \includegraphics[width=10cm]{compression/hg}
  \caption{Plot of how well the human genome of \cref{sec:intro:test-data} compresses compared to a random string.}
  \label{fig:compression:quality:hg}
\end{figure}

\subsection{Running time}
The running time of the algorithm should be bounded by $O(n\log{n})$ since the LZ-factorization can be found in linear time, and the concatenation and search steps all can be done in $O(\log{n} |f_i|)$ for each factor $f_i$. Since $f_1 + \dots + f_k = n$ the running time follows.

\Cref{fig:compression:runningtime} shows the normalized running time for constructing the compressed SLP. It can be seen that the upper line seems to converge to a constant, which indicates that the theoretical bound is also satisfied in practice.

An interesting thing to notice, is that the constant in the running time for the random strings is higher than for the other strings. Before actually concluding that the theoretical running time is satisfied, it is needed to argue that there does not exist another type of input that behaves even worse than the random strings running time wise.

The uniform random input will expected make the fragments that are reusable in the SLP be as small as possible. Therefore the random input will lead to many small insertions in the SLP, which will be balanced by rotations. This big amount of rotations is probably the cause of the increased constant, and since it is expected that no input will behave worse than the uniformly random distributed. This combined with \cref{fig:compression:runningtime} indicates that the running time is satisfied.

This also means that this step should not be the bottleneck of the algorithm, since it is asymptotically dominated by the other part of the algorithm doing the actual edit distance computation.
%
\begin{figure}[h!]
  \centering
  \includegraphics[width=10cm]{compression/runningtime}
  \caption{Normalized running time for construction of compressed SLP.}
  \label{fig:compression:runningtime}
\end{figure}

\chapter{SLP based edit distance algorithm}
\label{ch:algorithm}

This chapter will describe and benchmark the algorithm for computing the edit distance of two input SLPs $\SLP{A}$ and $\SLP{B}$.

\section{Preliminaries}
\label{sec:algorithm:preliminaries}
In order to describe the algorithm and argue its correctness, some notation will be introduced.

It has been chosen that all matrices used in this report are $0$-indexed. That is, an entry in a matrix $A$ of size $m \times n$ is indexed by $A(i, j)$ where $i \in \{0, \dots, m - 1\}, j \in \{0, \dots, n - 1\}$. Note that this is a different choice than the referenced literature describing the algorithm (most notably \cite{Tiskin:2010:FDM:1873601.1873704} and \cite{DBLP:journals/corr/abs-0707-3619}), who for convenience use odd half-integers to index some of the matrices. This approach was not chosen in order to make the calculations in this report closer to the actual \texttt{C++} implementation.

Almost all matrices considered in this report will be square matrices, ie. of size $n \times n$ for some $n \in \mathbb{N}$. Therefore all following definitions will only be given for square matrices, but most of them could easily be generalized to matrices of arbitrary dimension.

\begin{mydef}
  A permutation matrix of size $n$ is a matrix over $\{0,1\}^{n \times n}$ where each row and each column contains exactly one $1$. That is, it can be seen as the identity matrix of size $n \times n$ where the rows (or columns) are permuted.
  It is assumed that the index of the column (row) containing the $1$ can be found in constant time from a given row (column).
\end{mydef}
The latter constraint on the query time of the $1$ entry is required to achieve the running time of some of the algorithms (ex \cref{lemma:simple-unit-monge-query-next}). The way permutation matrices are represented in the implementation, is that the row and column permutation are specified using $2n = O(n)$ machine words. This directly gives the constant time query to the $1$ entry of a given row/column.

The algorithm stores a lot of permutation matrices in memory. It is therefore interesting to try to minimize the space consumption of these matrices. However, it turns out that the space consumption of $O(n)$ machine words is asymptotically optimal: A permutation matrix of size $n$ can express $n!$ different permutation matrices (since there are $n!$ different permutations of the rows of the identity matrix), hence at least $\Omega(\log{n!})$ bits are needed to represent a permutation matrix. By Stirling's approximation it is known that
\[
  \ln{n!} = n \ln{n} - n + O(\ln{n}),
\]
hence $\Omega(\log{n!}) = \Omega(n \log{n})$ bits are required to represent the permutation matrix, which is also what is used in the row/column-permutation representation.
%
\begin{mydef}
  The distribution matrix $A^{\Sigma}$ of a $n \times n$ matrix $A$ is a $(n + 1) \times (n + 1)$ matrix defined as
  \[
    A^{\Sigma}(i, j) = \sum_{\substack{\hat{i} \in \{i, \dots, n - 1\} \\ \hat{j} \in \{0, \dots, j - 1\}}} {A(\hat{i}, \hat{j})},
  \]
  or informally the sum of all entries in the sub-matrix to the left and below of a given entry $(i, j)$.
\end{mydef}
%
\begin{mydef}
  The density matrix $A^{\Box}$ of a $(n + 1) \times (n + 1)$ matrix is a $n \times n$ matrix defined as
  \[
    A^{\Box}(i, j) = A(i, j + 1) - A(i, j) - A(i + 1, j + 1) + A(i + 1, j)
  \]
  for all $i, j \in \{ 0, \dots, n - 1 \}$. \todo{Graphical representation?}
\end{mydef}%
The distribution and density operators always have the useful relation stated by the following claim.
\begin{claim}[\reftiskin{p. 1288}{remarked}]
  \label{claim:sigma-box-identity}
  For any matrix $A$ of size $n \times n$ it is the case that
  \[
    (A^{\Sigma})^{\Box}(i, j) = A(i, j)
  \]
  \begin{proof}
    The proof follows directly by unfolding definitions:\\
    \scalebox{0.9}{\parbox{.5\linewidth}{%
    \begin{align*}
      (A^{\Sigma})^\Box(i, j) &= A^{\Sigma}(i, j + 1) - A^{\Sigma}(i, j) - A^{\Sigma}(i + 1, j + 1) + A^{\Sigma}(i + 1, j) \\
        &= \sum_{\substack{\hat{i} \in \{i, \dots, n - 1\} \\ \hat{j} \in \{0, \dots, j\}}} {A(\hat{i}, \hat{j})}
         - \sum_{\substack{\hat{i} \in \{i, \dots, n - 1\} \\ \hat{j} \in \{0, \dots, j - 1\}}} {A(\hat{i}, \hat{j})}
         - \sum_{\substack{\hat{i} \in \{i + 1, \dots, n - 1\} \\ \hat{j} \in \{0, \dots, j\}}} {A(\hat{i}, \hat{j})}
         + \sum_{\substack{\hat{i} \in \{i + 1, \dots, n - 1\} \\ \hat{j} \in \{0, \dots, j - 1\}}} {A(\hat{i}, \hat{j})} \\
        &= A(i, j)
    \end{align*}
    }}\\
  \end{proof}
\end{claim}
If the operators are applied in the opposite order, it will not always describe a identity transformation. A class of matrices with this property is defined.
\begin{mydef}[\reftiskin{p. 1288}{Definition 2.1}]
  \label{def:simple}
  A matrix $A$ is called simple if $(A^{\Box})^\Sigma = A$.
\end{mydef}
The set of simple matrices can be classified by the following claim.
\begin{claim}[\reftiskin{p. 1288}{remark without proof}]
  \label{claim:simple-matrix-characterization}
  A matrix $A$ of size $(n + 1) \times (n + 1)$ is simple iff $\forall i,j: A(n, j) = A(i, 0) = 0$, ie. iff the leftmost column and bottom row have all zero entries.
  \begin{proof}
    The proof follows almost directly by unfolding definitions. For all $i, j \in \{ 0, \dots, n \}$ it follows that%
    \begin{align*}
      (A^{\Box})^{\Sigma}(i, j) &= \left( \left\{ A(\hat{i}, \hat{j} + 1) - A(\hat{i}, \hat{j}) - A(\hat{i} + 1, \hat{j} + 1) + A(\hat{i} + 1, \hat{j}) \right\}_{\hat{i}, \hat{j}} \right)^{\Sigma}(i, j) \\
        &= \sum_{\substack{\hat{i} \in \{i, \dots, n - 1\} \\ \hat{j} \in \{0, \dots, j - 1\}}} { A(\hat{i}, \hat{j} + 1) - A(\hat{i}, \hat{j}) - A(\hat{i} + 1, \hat{j} + 1) + A(\hat{i} + 1, \hat{j}) } .
    \end{align*}%
    By splitting the sum on every term, changing indexes in the sum and using the definition of the $\Sigma$-operator, it is found that
    \begin{align*}
      (A^{\Box})^{\Sigma}(i, j) &=
        \begin{aligned}
          &A^{\Sigma}(i, j + 1) - \sum_{\mathclap{\hat{j} \in \{0, \dots, j\}}} A(n, \hat{j}) - \sum_{\mathclap{\hat{i} \in \{i, \dots, n\}}} {A(\hat{i}, 0)} \\
          - &\left( A^{\Sigma}(i, j) - \sum_{\mathclap{\hat{j} \in \{0, \dots, j - 1\}}} A(n, \hat{j}) \right) \\
          - &\left( A^{\Sigma}(i + 1, j + 1) - \sum_{\mathclap{\hat{i} \in \{i + 1, \dots, n\}}} A(\hat{i}, 0) \right) \\
          + &A^{\Sigma}(i + 1, j)
        \end{aligned} \\
        &= \sum_{\hat{i} \in \{i, \dots, n\}} A(\hat{i}, j) - \sum_{\hat{i} \in \{i + 1, \dots, n\}} A(\hat{i}, j) - A(n, j) - A(i, 0) \\
        &= A(i, j) - A(n, j) - A(i, 0).
    \end{align*}
    Therefore a matrix is simple iff $\forall i, j: A(i, 0) = -A(n, j)$ iff $\forall i, j: A(i, 0) = A(n, j) = 0$.
    \todo{Should the last iff be more detailed?}
  \end{proof}
\end{claim}
%
\begin{mydef}[\reftiskin{p. 1288}{Definition 2.4}]
  A matrix $A$ is called Monge if $A^{\Box}$ is non-negative.
\end{mydef}
\begin{mydef}[\reftiskin{p. 1288}{Definition 2.6}]
  \label{def:unit-monge}
  A square matrix $A$ is called unit-Monge if $A^{\Box}$ is a permutation matrix.
\end{mydef}
It follows clearly that any unit-Monge matrix is also Monge since the identity matrix has no negative entries. The following lemma characterizes the unit-Monge matrices.
\begin{lemma}[\refbook{p. 10}{Lemma 2.11}]
  \label{lemma:unit-monge-characterization}
  Let $A \in \mathbb{N}^{(n + 1) \times (n + 1)}$ be a simple Monge matrix. Then $A$ is unit-Monge iff
  \[
    A(i, n) - A(i + 1, n) = 1 \quad \text{ and } \quad A(0, j + 1) - A(0, j) = 1
  \]
  for all $i, j \in \{0, \dots, n - 1\}$.
  \begin{proof}
    Since all entries of $A$ are natural numbers, the entries of $A^{\Box}$ are all integers directly by definition. Also, since $A$ is Monge, it is known that $A^{\Box}(i, j) \geq 0$ for all $i, j \in \{0, \dots, n - 1\}$. Therefore it can be noticed that $A^{\Box}$ is a permutation matrix iff
    \[
      \sum_{\hat{j} \in \{0, \dots, n - 1\}} A^{\Box}(i, \hat{j}) = 1
      \quad \text{ and } \quad
      \sum_{\hat{i} \in \{0, \dots, n - 1\}} A^{\Box}(\hat{i}, j) = 1
    \]
    for all $i, j \in \{0, \dots, n - 1\}$. Both of the sums can be written out to find that for all $i, j \in \{0, \dots, n - 1\}$:
    \begin{align*}
      \sum_{\hat{j} \in \{0, \dots, n - 1\}} A^{\Box}(i, \hat{j})
        &= \sum_{\hat{j} \in \{0, \dots, n - 1\}} \left( A(i, \hat{j} + 1) - A(i, \hat{j}) - A(i + 1, \hat{j} + 1) + A(i + 1, \hat {j}) \right) \\
        &= A(i, n) - A(i, 0) - A(i + 1, n) + A(i + 1, 0) \\
        \text{\cref{claim:simple-matrix-characterization}} &= A(i, n) - A(i + 1, n)
    \end{align*}
    \begin{align*}
      \sum_{\hat{i} \in \{0, \dots, n - 1\}} A^{\Box}(\hat{i}, j)
        &= \sum_{\hat{i} \in \{0, \dots, n - 1\}} \left( A(\hat{i}, j + 1) - A(\hat{i}, j) - A(\hat{i} + 1, j + 1) + A(\hat{i} + 1, j) \right) \\
        &= A(0, j + 1) - A(n, j + 1) - A(0, j) + A(n, j) \\
        \text{\cref{claim:simple-matrix-characterization}} &= A(0, j + 1) - A(0, j)
    \end{align*}
    Therefore $A^{\Box}$ is a permutation matrix (i.e. $A$ is unit-Monge) iff
    \[
      A(i, n) - A(i + 1, n) = 1 \quad \text{ and } \quad A(0, j + 1) - A(0, j) = 1
    \]
    for all $i, j \in \{0, \dots, n - 1\}$.
  \end{proof}
\end{lemma}
Notice that if a matrix $A$ is both simple and unit-Monge (from now on denoted simple unit-Monge), then from the unit-Monge property, there exists some permutation matrix $P$ such that $P = A^{\Box}$, and by applying the $\Sigma$-operator and using the simple property, it is found that $P^\Sigma = (A^{\Box})^{\Sigma} = A$. Therefore a simple unit-Monge matrix can be represented in linear space by a permutation matrix.

An important observation required for the algorithm described in \cref{sec:algorithm:min-mult-two-unit-monge} is the following.
\begin{lemma}[\reftiskin{p. 1288}{Theorem 2.1}]
  \label{lemma:simple-unit-monge-query-next}
  Given a permutation matrix of size $n$, then given the value of $P^{\Sigma}(i, j)$ for some $i, j \in \{0, \dots, n\}$, the adjacent entries $P^{\Sigma}(i \pm 1, j), P^{\Sigma}(i, j \pm 1)$ where they exist, can be computed in constant time.
  \begin{proof}
    Because a permutation matrix $P$ only contains a single $1$ in any row or column, the value of adjacent entries in $P^{\Sigma}$ can only differ by either $-1$, $0$ or $1$. Determining which is the case is easy by looking at the query direction and in which column (row) the $1$ entry in $P$ is located.

    For example, to query the entry to the right of the given entry, it is easily seen that
    \begin{align*}
      P^{\Sigma}(i, j + 1) &= \sum_{\substack{\hat{i} \in \{i, \dots, n - 1\} \\ \hat{j} \in \{0, \dots, j\}}} P(\hat{i}, \hat{j})
        = \sum_{\substack{\hat{i} \in \{i, \dots, n - 1\} \\ \hat{j} \in \{0, \dots, j - 1\}}} P(\hat{i}, \hat{j}) + \sum_{\hat{i} \in \{i, \dots, n - 1\}} P(\hat{i}, j) \\
        &= P^{\Sigma}(i, j) + \begin{cases}
            1  & \text{if position of $1$ in $j$'th row of $P \geq i$} \\
            0  & \text{o/w}
          \end{cases},
    \end{align*}
    where the latter can be calculated in $O(1)$ time by assumption of constant query time on permutation matrices.
  \end{proof}
\end{lemma}

\begin{mydef}[\reftiskin{p. 1289}{Definition 3.1}]
  \label{def:minimum-distance-product}
  Given two matrices $A, B$ both of size $n \times n$ the minimum distance product of the two matrices is
  \[
    (A \odot B)(i, j) = \min_{k \in \{ 0, \dots, n - 1 \}} \left( A(i, k) + B(k, j) \right)
  \]
  The maximum distance product is defined analogously but where the $\min$ is replaced by $\max$.
\end{mydef}
The algorithm crucially relies on the following claim, which implies the minimum distance product of two simple unit-Monge can be succinctly represented by a permutation matrix.
\begin{claim}[\refbook{p. 16}{Theorem 3.4}]
  \label{claim:unit-monge-min-prod-closed}
  Given two simple unit-Monge matrices $A, B$, then their minimum distance product $A \odot B$ is also simple unit-Monge.
  \begin{proof}
    \todo{Give the proof, of simply reference proof in paper?}
  \end{proof}
\end{claim}
\Cref{claim:unit-monge-min-prod-closed} gives rise to the following notation that will be useful later.
\begin{mydef}[\reftiskin{p. 1289}{Definition 3.2}]
  Given permutation matrices $P_A$ and $P_B$, then $P_A \boxdot P_B = P_C$ if $P_C^{\Sigma} = P_A^{\Sigma} \odot P_B^{\Sigma}$.
\end{mydef}

\section{Algorithm}
The algorithm follows the steps of \todo{ref}, however instead of directly computing the edit distance between the two input SLPs, the longest common subsequence (LCS) between the two SLPs will be computed. This is done because it turns out that it is easier to succinctly represent the DIST tables for the LCS problem, in such a way that computations can efficiently be performed.

\begin{mydef}
  The longest common subsequence (LCS) problem is given two input strings (or SLPs) $\str{A}{0}{m}$ and $\str{B}{0}{n}$. The result is the longest substring occurring in both strings.
\end{mydef}

The LCS problem can up to a simple transformation of the result be seen as the edit distance, but where substitutions are not allowed. This transformation is given by
\[
  \EditDist'(A, B) = n + m - 2 \cdot lcs(A, B)
\]
and follows since the above expression counts the number of unmatched symbols, which is the only cost in the edit distance when substitutions are not allowed. As described later in this section, it is possible to simulate the substitution operation by modifying the input strings.

The simple algorithm for calculating the LCS between two strings is almost the same as the simple algorithm for computing the edit distance given in \cref{sec:intro:simple}. Therefore, the same terminology with the dynamic programming grid is still valid. From this point on the dynamic programming grid will always refer to the one in the LCS problem. 

\begin{mydef}
  \label{defn:dist-table}
  A $\DIST$ table between two strings $\str{A'}{0}{m}$ and $\str{B'}{0}{n}$ stores for all prefix-suffix pairs of $A'$ and $B'$ the longest common substring between them. More specifically for input and outputs $i, o \in \{0, \dots, m + n\}$ where $i \leq o + m$ and $o \leq i + n$ the $\DIST$ tables stores
  \[
    \DIST_{A',B'}[i, o] = \left\{
    \begin{array}{ll}
      lcs(\substr{A'}{m - i}{m}, \substr{B'}{0}{o})             & \text{if } i < m, o < n \\
      lcs(A', \substr{B'}{i - m}{o})                            & \text{if } i \geq m, o < n \\
      lcs(\substr{A'}{m - i}{m - (o - n)}, B')                  & \text{if } i < m, o \geq n \\
      lcs(\substr{A'}{0}{m - (o - n)}, \substr{B'}{i - m}{n})   & \text{if } i \geq m, o \geq n
    \end{array}
  \right. .
  \]

  More importantly intuition wise, the $\DIST$ table can be viewed as a hollow box in the dynamical programming grid, where input and output are numbered along the border from the bottom left corner. An illustration of a $\DIST$ table is given in \cref{fig:defn-dist-table}.

  \begin{figure}[h!]
    \centering
    \begin{subfigure}{0.45\textwidth}
      \includegraphics[width=\textwidth]{images/grid-dist}
      \caption{Numbering of inputs and outputs of a $\DIST$-table.}
    \end{subfigure}
    %
    \begin{subfigure}{0.45\textwidth}
      \includegraphics[width=\textwidth]{images/grid-dist-2}
      \caption{How $\DIST$ tables fit together.}
    \end{subfigure}
    \caption{Illustration of $\DIST$ tables.}
    \label{fig:defn-dist-table}
  \end{figure}
\end{mydef}

In order for this technique to work, it is required that the edit distance reduces to the LCS problem with only a constant overhead in time and space consumption. The reduction is given in \cite[p. 71]{DBLP:journals/corr/abs-0707-3619} where it is called the \textit{blow-up technique}.

\paragraph{Blow-up technique}
The idea is to "blow-up" the input strings by a constant fraction by inserting special characters, which in a suitable way will relate the length of the LCS of the transformed strings to the edit distance of the original input strings.

The transformation is as follows: All symbols $a \in \Sigma$ in the inputs strings are transformed to $a\$$, where $\$ \not\in \Sigma$. Denote the transformed strings by $A^{\$}$ and $B^{\$}$. In order to argue that this works, one can consider the possible edit distance operations and show how these can be emulated by computing $lcs$ on the transformed strings. The cases are illustrated on \cref{fig:blow-up:edit-dist-cases}.
%
\begin{figure}[h!]
  \centering
  \begin{subfigure}{0.8\textwidth}
    \includegraphics[width=\textwidth]{images/blow-up-edit-dist-cases}
    \caption{Cases that corresponds to edit distance operations.}
    \label{fig:blow-up:edit-dist-cases}
  \end{subfigure}
  %
  \begin{subfigure}{0.8\textwidth}
    \centering
    \includegraphics[width=0.15\textwidth]{images/blow-up-not-possible-case}
    \caption{Case that is not possible in longest common subsequence.}
    \label{fig:blow-up:not-possible}
  \end{subfigure}
  \caption{Possible cases for aligning two transformed strings in the longest common subsequence world.}
\end{figure}%
%
Notice that it is not possible to create a longest common subsequence using other matching strategies than the ones shown in \cref{fig:blow-up:edit-dist-cases}. The only one that is omitted is the matching shown in \cref{fig:blow-up:not-possible}, but this can always be improved by matching the $\$$'s and hence it can not be the longest common substring.

Due to the scorings of the possible cases in \cref{fig:blow-up:edit-dist-cases}, it is easy to verify that
\[
  \EditDist(A, B) = n + m - lcs(A^{\$}, B^{\$}),
\]
by simply considering how the alignment relates for the different cases.

This approach results in the strings becoming a factor of $2$ longer. For a simple $O(N^2)$ implementation of the LCS algorithm, this would result in a factor of $2^2 = 4 = O(1)$ slow down. However, for LCS working on compressed strings, the penalty might not be as severe, since the blown up strings compresses well. This follows from, and also gives an algorithm for performing the blow-up: Every terminal in the SLP (of which there are $|\Sigma|$) is replaced by a non-terminal producing the original terminal and a new terminal producing the $\$$ symbol. This results in only $|\Sigma| + 1$ new productions being added to the SLP encoding of the string.

\subsection{Selection of productions}
\label{sec:algorithm:select-productions}
In order to be able to save time, the idea is to pre-compute DIST tables based on the productions in the SLPs, such that reused productions means that less DIST tables have to be precomputed. Obviously there is not enough time to compute DIST tables for all productions, as this would directly compute the longest common subsequence by looking at the result in the root node of the SLPs. Instead, the SLPs will be chopped into blocks deriving strings of length $O(x)$ for some $x \in \mathbb{N}$, and then the DIST tables will precomputed for these blocks.

Both of the SLPs will be blocked in the same way. The blocking process consists of two steps. First what will be known as \textit{key productions} are selected and then productions in between the key productions are selected.

The key productions are identified by traversing the unfolded SLP bottom up. If the string derived by the current production has length less than $x$ it is annotated with an associated string, which in this case is the string it derives\footnote{In practice the actual associated string is not needed, but only the length of the associated string. Hence in the code only the length of the associated string is stored.}. If the length is greater than $x$ and both its children have length less than x, then the production is marked as a key production and its associated string is also set to the string it derives. Otherwise no other action is taken.

All the selected key productions covers some part of the string. In order to cover the rest of the input strings extra productions needs to be added, which are called the second type of productions. In order to find these, the path between two adjacent key productions is traversed. This is done by first finding the lowest common ancestor of the two key productions, and then traversing the paths from both of the key productions to the lowest common ancestor bottom up. Notice that all productions hanging off the traversed paths (\todo{figure}) must all have derived strings with length less than $x$ since they would be marked as key productions otherwise which is a contradiction with the current choice of the key productions.

When traversing up the left path, the strings derived by the productions hanging of to the right are associated to the production on the path. These strings are merged bottom up in such a way that they have length less than $2x$. Every time a string of length greater than $x$ is generated, the production generating it is selected, and its associated string is the merged string on the path. This process is done analogously for the right path.

There is a special case with the beginning and end of the string, as there are not necessarily a key production spanning the first and the last part of the string. However, these cases be accommodated by the path traversing algorithm, where there is only one left path for the beginning and one right path for the end.

By this process, all the selected productions will span the entire input string and they are all associated with strings of length $O(x)$.

Due to the bottom up nature of the selection process, a repeated production in two parts of the string will result in the same selected productions. This is very important in obtaining the desired speedup, since DIST tables are precomputed for all the selected productions. Therefore, reducing the number of different selected productions yields a time improvement when building the DIST repository.

One technical detail to notice for the above approach to work directly, is that a production is only associated with one unique string, even through a production may occur several times in the unfolded SLP. Due to the bottom up selection, this is clear for the key productions. For the second type it follows since the same key productions are selected in different parts of the strings, resulting in the paths between the key productions being unique, so that they always produces the same result.

\subsection{Building DIST repository}
\label{sec:algorithm:building-DISTs-overview}
In the previous step the productions for which the DIST repository should be build for. This section will focus on how the DIST repository is constructed. More specifically the DIST repository contains a DIST table for every pair of selected in the two SLPs. In other words, let $P_\mathcal{A}$ and $P_\mathcal{B}$ denotes the selected productions in respectively the SLPs $\mathcal{A}$ and $\mathcal{B}$. Then the dist repository contains a DIST table $\DIST_{a,b}$ for all $a \in P_\mathcal{A}, b \in P_{\mathcal{B}}$. The $\DIST$ tables was only defined for strings in \cref{defn:dist-table}, however it is easy to generalize to productions of a SLP by using the associated string of the production.

The $\DIST$ tables will be constructed by means of recursively merging $\DIST$ tables for sub-productions of the selected productions. The base case for a pair of terminals is not explicitly described in this section, but is deferred to when the concrete representation of the $\DIST$ tables have been described \todo{ref}.

The previously described production selection process divides the the selected productions into two types. The following describes how to handle the two cases.
\begin{description}
  \item[Key productions] A key production is always associated with the string that the production derives. Therefore assume the DIST table for two key productions $a \in P_{\mathcal{A}}$ and $b \in P_{\mathcal{B}}$ should be built. Name the children of the two productions by $a_L, a_R$ and $b_L, b_R$ as illustrated on \cref{fig:distrepo-key}.
  
  \begin{figure}[h!]
    \centering
    \includegraphics[width=8cm]{images/distrepo-key}
    \caption{Illustration of a key production.}
    \label{fig:distrepo-key}
  \end{figure}
  
  The DIST table is constructed by first recursively constructing the DIST tables for all combinations of the children, ie. $(a_L, b_L), (a_L, b_R), (a_R, b_L)$ and $(a_R, b_R)$. Then the DIST from $a$ and $b$ can be obtained by merging as shown in Figure~\ref{fig:compression:dist:merge}. For now merging of $\DIST$ tables will be assumed to be a black box. The details of the merging process will be given in \cref{sec:merging-dists}.
  \begin{figure}[h!]
    \centering
    \includegraphics[width=10cm]{images/dist_merge}
    \caption{Merging phases of DIST tables from the children to obtain the DIST for the concatenated strings.}
    \label{fig:compression:dist:merge}
  \end{figure}

  \item[Second type of selected productions] For this type of productions the associated string is not the string that the production derives. Instead the associated string is the right (or left) child, possibly combined with the right (or left) child of its left (or right) child. A graphical illustration is given on \cref{fig:distrepo-2nd} where the production is on a left path. The right path case is symmetric and is not included.

  \begin{figure}[h!]
    \centering
    \includegraphics[width=5cm]{images/distrepo-2nd}
    \caption{Illustration of how a production of the second type is handled.}
    \label{fig:distrepo-2nd}
  \end{figure}

  By adopting the notation of $a_L$ and $a_R$ from \cref{fig:distrepo-key}, and using the appropriate definitions of $b_L, b_R$ from the type of the $b$ production, the merging process of \cref{fig:compression:dist:merge} gives the desired result.
\end{description}
Since the number of productions is bounded by $O(n^2)$, memorization can be used in combination with the above procedure to construct the dist repository in time $O(n^2 M(x))$, where $M(x)$ denotes the time for merging two DIST tables. As will be seen in \cref{sec:merging-dists}, the merging can be done in time $M(x) = O(n\log{n})$, leading to a total running time of $O(n^2x\log{x})$ for computing the DIST repository.

\subsection{Filling out the dynamical programming table}
\label{sec:algorithm:filling-grid-overview}
The dynamical programming table is filled from top to bottom, left to right. This is done by applying the $\DIST$ tables one by one using the selected productions. This is done by keeping two columns of the table in memory and $O(x)$ of a row. These specifies the inputs $I$ to the $\DIST$ table, and by using \cref{defn:dist-table} it is easy to see that the outputs can then be computed by
\begin{equation}
  \label{eqn:dist-application}
  O[j] = \min_i \{ I[i] + \DIST[i, j] \},
\end{equation}
since the $\DIST$ table stores the shortest path from input $i$ to output $j$. Evaluating the above formula directly for every input takes $O(x^2)$ time, which is too much.

However, by representing the $\DIST$ tables in a clever way, the time required to evaluate \cref{eqn:dist-application} can be reduced to $O(x)$. This algorithm is described in \cref{sec:applying-a-dist-table}. Therefore, the total time for filling out the grid becomes $O\left( \frac{N^2}{x^2}x \right) = O\left( \frac{N^2}{x} \right)$.

\subsection{Combining the two steps}
In order to reduce the running time, the idea is to asymptotically balance the time used to compute the $\DIST$ repository and to fill out the grid. In order to do this, define $f = \frac{N}{n}$, which is the compression ratio of the input strings.

The size of the $\DIST$ tables $x$ is defined to be $x = \frac{f}{\sqrt{\log{f}}}$. This choice of $x$ results in the $\DIST$ repository taking
\begin{align*}
  O(n^2 x\log{x})
    &= O\left( n^2 \frac{f}{\sqrt{\log{f}}} \log\left( \frac{f}{\sqrt{\log{f}}} \right) \right)
    = O\left( Nn \frac{1}{\sqrt{\log{\frac{N}{n}}}} \log\left( \frac{N/n}{\sqrt{\log{\frac{N}{n}}}} \right) \right) \\
    &= O\left( Nn \frac{1}{\sqrt{\log{\frac{N}{n}}}} \left( \log{\frac{N}{n}} - \log{\sqrt{\log{\frac{N}{n}}}} \right) \right)
    = O\left( Nn\sqrt{\log{\frac{N}{n}}} \right)
\end{align*}
time to build. The grid can be filled in time
\[
  O\left(\frac{N^2}{x}\right)
    = O\left( \frac{N^2}{\frac{f}{\sqrt{\log{f}}}} \right)
    = O\left( \frac{N^2 \sqrt{\log{f}}}{f} \right)
    = O\left( Nn\sqrt{\log{\frac{N}{n}}} \right).
\]
Since the SLP construction algorithm runs in time $O(N\log{N})$ which will always be less, the combined running time for the entire combined algorithm becomes $O\left( Nn\sqrt{\log{\frac{N}{n}}} \right)$. In the big-O notation this running time is always superior to $O(N^2)$ which is the time required by the simple algorithm.

\section{Representation of $\DIST$s}
As mentioned in the previous section, representing the $\DIST$s in a succinct way is crucial in order to obtain the desired running time.

Given two strings $A$ and $B$, a succinct representation of the LCS $\DIST$ tables should be constructed. By \cref{defn:dist-table} it is known that
\[
  \DIST_{A,B}[i, j] = \left\{
    \begin{array}{ll}
      lcs(\substr{A}{m - i}{m}, \substr{B}{0}{j})             & \text{if } i < m, j < n \\
      lcs(A, \substr{B}{i - m}{j})                            & \text{if } i \geq m, j < n \\
      lcs(\substr{A}{m - i}{m - (j - n)}, B)                  & \text{if } i < m, j \geq n \\
      lcs(\substr{A}{0}{m - (j - n)}, \substr{B}{i - m}{n})   & \text{if } i \geq m, j \geq n
    \end{array}
  \right. ,
\]
that is, the $\DIST$ can be summarized as the LCS between all prefix-suffix pairs of the two strings $A$ and $B$.
Notice that a $\DIST$ table is not well-defined for $i > j + m$ and $j > i + n$ since it corresponds to non-existing paths in the dynamic programming table. In all following calculations and statements these invalid index combinations should simply be considered as statements without any information.

A different way of describing the LCS between these prefix-suffix pairs, is by the following definition.

\begin{mydef}[\refbook{p.-48}{Definition 4.8}]
  \label{def:H-table}
  Let $\str{A}{0}{m}$ and $\str{B}{0}{n}$ be strings, then for $i, j \in \{ 0, \dots, m + n \}$
  \[
    H_{A,B}(i, j) := \begin{cases}
                        j - (i - m)                         & \text{if j < i - m} \\
                        lcs(A, \substr{?^mB?^m}{i}{j + m})  & \text{o/w}
                      \end{cases}
  \]
  where $? \not\in \Sigma$ denotes a wildcard character matching all other symbols.
\end{mydef}
The first case in the $H_{A,B}$ definition is introduced because of the entries in the corresponding $\DIST$ table that are not well-defined. The $H$-table entries corresponding to these undefined $\DIST$ entries are simply chosen to make everything work out as they should in order to make the $H$-tables easy representable. In almost all the following calculations it has been chosen not to include this case as it is straight forward that the choice will work. This is due to the definition of the $\Sigma$-operator and that unit-Monge matrices are closed under minimum distance multiplication.

The precise correspondence between the $H$-table of \cref{def:H-table} and the $\DIST$ tables, is given by the following lemma.
\begin{lemma}[\refbook{p.-48}{mentioned but not proved}]
  \label{lemma:dist-H-relation}
  \[
  DIST[i, j] = H(i, j) + \left\{
    \begin{array}{ll}
      i - m             & \text{if } i < m, j < n \\
      0                 & \text{if } i \geq m, j < n \\
      (i - m) + (n - j) & \text{if } i < m, j \geq n \\
      n - j             & \text{if } i \geq m, j \geq n
    \end{array}
  \right.
  \]
  \begin{proof}
    The lemma follows almost directly by the definition of the DIST, $H$-table and $lcs$ function, by splitting the claim up into the four different cases of the statement.
    \begin{description}
      \item[\circled{1}] For $i \in \{0, \dots, m\}, j \in \{0, \dots, n\}$,
       \begin{align*}
         H(i, j) &= lcs(A, \substr{?^mB?^m}{i}{j + m}) = lcs(A, ?^{m - i}(\substr{B}{0}{j})) \\
                 &= (m - i) + lcs(\substr{A}{m - i}{m}, \substr{B}{0}{j}) = \DIST[i, j] + m - i
       \end{align*}

      \item[\circled{2}] For $i \in \{m, \dots, m + n\}, j \in \{0, \dots, n\}$,
        \[
          H(i, j) = lcs(A, \substr{?^mB?^m}{i}{j + m}) = lcs(A, \substr{B}{i - m}{j}) = \DIST[i, j]
        \]

      \item[\circled{3}] For $i \in \{0, \dots, m\}, j \in \{n, \dots, n + m\}$,
        \begin{align*}
          H(i, j) &= lcs(A, \substr{?^mB?^m}{i}{j + m}) = lcs(A, ?^{m - i}B?^{j - n}) \\
                  &= (m - i) + (j - n) + lcs(\substr{A}{m - i}{m - (j - n)}, B) \\
                  & = (m - i) + (j - n) + \DIST[i, j]
        \end{align*}

      \item[\circled{4}] For $i \in \{m, \dots, m + n\}, j \in \{n, \dots, n + m\}$,
        \begin{align*}
          H(i, j) &= lcs(A, \substr{?^mB?^m}{i}{j + m}) = lcs(A, \substr{B}{i - m}{n}?^{j - n}) \\
                  &= (j - n) + lcs(\substr{A}{0}{m - (j - n)}, \substr{B}{i - m}{n}) = (j - n) + \DIST[i, j]
        \end{align*}
    \end{description}
  \end{proof}
\end{lemma}
An important property of the $H$-table is that it is almost simple unit-Monge.
\begin{lemma}[\refbook{p.-49}{Theorem 4.10}]
  \label{lemma:H-permutation-representation}
  Given a $H$-table $H$ satisfying \cref{def:H-table}, there exists a permutation matrix $P$ of size $m + n - 1$ such that
  \[
    H(i, j) = j - (i - m) - P^{\Sigma}(i, j).
  \]
  Therefore, the permutation matrix $P$, which can be represented in linear space, encodes the $H$-table, and thereby also the $\DIST$-table.

  \begin{proof}
    Define the following matrix to be a small pertubation of the $H$-table
    \[
      N(i, j) := j - (i - m) - H(i, j).
    \]
    We will prove that $N$ is simple unit-Monge by which the statement follows. First it will be shown that $N$ is Monge, that is $N^{\Box} \geq 0$. From the definition of $N$ and $\Box$ it is needed to show that
    \begin{align*}
      &H(i, j + 1) - H(i, j) - H(i + 1, j + 1) + H(i + 1, j) = H^{\Box}(i, j) \leq 0 \\
      \iff &H(i, j + 1) + H(i + 1, j) \leq H(i, j) + H(i + 1, j + 1)
    \end{align*}
    for all $i,j$. By \cref{def:H-table} the adjacent entries of the $H$-table in the above equation correspond to the LCS of a fixed string with a string where a character is possibly extended in both ends. The situation is illustrated in \cref{fig:H-permutation-representation:monge} where the top of the illustration corresponds to the left hand side of the inequality and the bottom corresponds to the right hand side.

    It is clear that the part between $i + 1$ and $j$ will always ensure some fixed contribution to all of the intervals. Now assume that the top contribution is strictly more than two times this fixed contribution. This can happen if it is possible to match an unmatched symbol by introducing cell $i$ or $j + 1$. It is easy to see that if this is the case, then it is easy to do the same for one (or both) of the bottom intervals. Therefore the bottom intervals will always match more than the top intervals, which gives the inequality and thereby the Monge property.
    \begin{figure}[h!]
      \centering
      \includegraphics[width=11cm]{images/monge-condition-illustration}
      \caption{Illustration of how the longest common subsequence extension relates to the Monge property.}
      \label{fig:H-permutation-representation:monge}
    \end{figure}

    Note the following properties about $N$:
    \begin{align*}
      N(i, 0) &= 0 - (i - m) - H(i, j)
        = (m - i) - \begin{cases}
          0 - (i - m)                    & \text{if } i > m \\
          lcs(A, \substr{?^mB?^m}{i}{m}) & \text{o/w}
        \end{cases} \\
        &= (m - i) - (m - i) = 0
    \end{align*}
    \begin{align*}
      N(m + n, j) &= j - ((m + n) - m) - H(i, j) \\
        &= j - n - \begin{cases}
          j - n                                   & \text{if } j < n \\
          lcs(A, \substr{?^mB?^m}{m + n}{j + m})  & \text{o/w}
        \end{cases} \\
        &= (j - n) - (j - n) = 0
    \end{align*}
    Therefore \cref{claim:simple-matrix-characterization} gives that $N$ is a simple matrix. Also note that the following equalities hold for $i,j \in \{0, \dots, m + n\}$:
    \begin{align*}
      N&(i, n + m) - N(i + 1, n + m) \\
        &= (n + m - (i - m) - H(i, n + m)) - (n + m - ((i + 1) - m)) - H(i + 1, n + m) \\
        &= 1 - H(i, n + m) + H(i + 1, n + m) \\
        &= 1 - lcs(A, \substr{?^mB?^m}{i}{2m+n}) + lcs(A, \substr{?^mB?^m}{i + 1}{2m+n}) \\
        &= 1 - m + m = 1
    \\ \\
      N&(0, j + 1) - N(0, j) \\
        &= (j + 1 - (0 - m) - H(i, j + 1)) - (j - (0 - m) - H(i, j)) \\
        &= 1 - H(0, j + 1) + H(0, j) \\
        &= 1 - lcs(A, \substr{?^mB?^m}{0}{j + 1 + m}) + lcs(A, \substr{?^mB?^m}{0}{j + m}) \\
        &= 1 - m + m = 1
    \end{align*}
    Since $N$ is simple Monge and have the above properties \cref{lemma:unit-monge-characterization} gives that $N$ is unit-Monge.
  \end{proof}
\end{lemma}
What remains is to ensure that the merge and apply operations operations on the $\DIST$ table can be computed efficiently using only the permutation matrix representation.

\subsection{Merging of $\DIST$ tables}
\label{sec:merging-dists}
The following definition will come in handy:
\begin{mydef}
  Let $P_{A'}$ and $P_{A''}$ be permutation matrices describing the $\DIST$ tables for the strings $\str{A'}{0}{m'}$ and $\str{A''}{0}{m''}$ respectively. Then define the following index change matrices
  \[
    IC'_{P_{A'}} = \begin{pmatrix}
      I^{m'' \times m''} & 0 \\
      0 & P_{A'}
    \end{pmatrix}, \quad
    IC''_{P_{A''}} = \begin{pmatrix}
      P_{A''} & 0 \\
      0 & I^{m' \times m'}
    \end{pmatrix}.
  \]
  Notice that both of these new matrices are also permutation matrices, and that they can be efficiently computed by a single scan of the original matrices.
\end{mydef}

The following convenient fact that will be useful later, follows directly from the definition of the $\Sigma$-operator:
\[
  {IC'_{P_{A'}}}^{\varsigma} = \begin{pmatrix}
    (I^{m'' \times m''})^{\varsigma} & \cdot \\
    0 & P_{A'}^{\varsigma}
  \end{pmatrix}, \quad
  {IC''_{P_{A''}}}^{\varsigma} = \begin{pmatrix}
    P_{A''}^{\varsigma} & \cdot \\
    0 & (I^{m' \times m'})^{\varsigma}
  \end{pmatrix},
\]
where $\varsigma$ denotes the $\Sigma$ operator, but where the last row and the first column are removed.

\subsubsection{Vertical merge}
Now it will be explained how a vertical merge between $A'$ and $A''$ can be found only giving the succinct representation of the DIST tables $\DIST_{A',B}$ and $\DIST_{A'',B}$. From a straight forward analysis of the LCS graph described by the $\DIST$ tables, it is found that:
\begin{claim}
  \label{claim:dist-vertical-merge}
  \item[\circled{1}]: For $i \leq m'', j \leq n + m''$:
    \[
      \DIST_{A'A'',B}[i, j] = \DIST_{A'',B}[i, j]
    \]
  \item[\circled{2}]: For $i \geq m'', j \geq n + m''$:
    \[
      \DIST_{A'A'',B}[i, j] = \DIST_{A',B}[i - m'', j - m''].
    \]
  \item[\circled{3}]: For $i \geq m'', j \leq n + m'$:
    \[
      \DIST[i, j] = \max_{k \in \{0, \dots, n\} } \left\{ \DIST_{A',B}[i - m'', k] + \DIST_{A'',B}[m'' + k, j] \right\}
    \]
  \begin{proof}
    Straight forward from the definition of the $\DIST$ tables in term of the dynamic programming table by the simple LCS algorithm. Illustration of the three different cases is shown in \cref{fig:dist-vertical-merge-cases}.

    \begin{figure}[h!]
      \centering
      \includegraphics[width=8cm]{images/dist-vertical-merge-cases}
      \caption{The tree possible cases when vertical merging two $\DIST$ tables.}
      \label{fig:dist-vertical-merge-cases}
    \end{figure}
  \end{proof}
\end{claim}
%
For case \circled{3}, the result from \cref{claim:dist-vertical-merge} can be further expanded to find
\begin{align*}
  \DIST_{A'A'',B}[i, j] &= \max_{k \in \{0, \dots, n\} } \left\{ \DIST_{A',B}[i - m'', k] + \DIST_{A'',B}[m'' + k, j] \right\} \\
              &=  \begin{aligned}
                    \max_k \left\{
                      H_{A',B}(i - m'', k) + \left\{
                        \begin{array}{ll}
                          i - m & \text{if } i - m'' < m' \\
                          0     & \text{o/w}
                        \end{array} \right. \right. \\
                      \left. +\,\,H_{A'',B}(m'' + k, j) + \left\{
                        \begin{array}{ll}
                          n - j & \text{if } j \geq n \\
                          0     & \text{o/w}
                        \end{array} \right.
                    \right\}
                 \end{aligned}\\
              &= \max_k \left\{ H_{A',B}(i - m'', k) + H_{A'',B}(m'' + k, j) \right\}
                  + \begin{cases}
                      (i - m) + (n - j)   & \text{if } i < m, j \geq n \\
                      i - m               & \text{if } i < m, j < n \\
                      n - j               & \text{if } i \geq m, j \geq n \\
                      0                   & \text{if } i \geq m, j < n
                    \end{cases}.
\end{align*}
Therefore \cref{lemma:dist-H-relation} now implies that $H_{A'A'',B}(i, j) = \max_k \left\{ H_{A',B}(i - m'', k) + H_{A'',B}(m'' + k, j) \right\}$. This expression can further be written out, to find
\begin{align*}
  H_{A'A'',B}(i, j) &= \max_k \left\{ H_{A',B}(i - m'', k) + H_{A'',B}(m'' + k, j) \right\} \\
                    &= \max_k \left\{ k - (i - m'' - m') - P_{A'}^{\Sigma}(i - m'', k) + j - (k + m'' - m'') - P_{A''}^{\Sigma}(k + m'', j) \right\} \\
                    &= j - (i - m) + \max_k \left\{ -\left( P_{A'}^{\Sigma}(i - m'', k) + P_{A''}^{\Sigma}(k + m'', j) \right)  \right\} \\
                    &= j - (i - m) - \min_k \left\{ P_{A'}^{\Sigma}(i - m'', k) + P_{A''}^{\Sigma}(k + m'', j) \right\}.
\end{align*}
Now \cref{lemma:H-permutation-representation} gives that
\begin{align*}
  P_{A'A''}^{\Sigma}(i, j) &= \min_k \left\{ P_{A'}^{\Sigma}(i - m'', k) + P_{A''}^{\Sigma}(k + m'', j) \right\} \\
                           &= \min_k \{ {IC'_{P_{A'}}}^{\Sigma}(i, k) + {IC''_{P_{A''}}}^{\Sigma}(k, j) \},
\end{align*}
which reduces this case to min-multiplying two simple unit-Monge matrices, which can be solved in time $O(n \log{n})$ by an algorithm given by \cite{Tiskin:2010:FDM:1873601.1873704} and described in \cref{sec:algorithm:min-mult-two-unit-monge}.

For case \circled{1} where $i \leq m'', j \leq n + m''$, the definition of the $\DIST$ table yields
\begin{align*}
  \DIST_{A'A'',B}[i, j] &= H_{A'A'',B}(i, j) + \left\{
    \begin{array}{ll}
      i - m             & \text{if } j < n \\
      (i - m) + (n - j) & \text{if } j \geq n
    \end{array}
  \right. \\
  &= (i - m) + j - (i - m) - P_{A'A'',B}^\Sigma(i, j) + \left\{
    \begin{array}{ll}
      0             & \text{if } j < n \\
      (n - j)       & \text{if } j \geq n
    \end{array}
  \right. \\
  &= j - P_{A'A'',B}(i, j) + \left\{
    \begin{array}{ll}
      0             & \text{if } j < n \\
      n - j         & \text{if } j \geq n
    \end{array}
  \right.
\end{align*}
By a similar analysis of the statement given by \cref{claim:dist-vertical-merge}, it is found that:
\begin{align*}
  \DIST_{A'A'',B}[i, j] &= \DIST_{A'',B}[i, j] = H_{A'',B}(i, j) + \left\{
    \begin{array}{ll}
      i - m''             & \text{if } j < n \\
      (i - m'') + (n - j) & \text{if } j \geq n
    \end{array}
  \right. \\
  &= \dots = j - P_{A'',B}^{\Sigma}(i, j) + \left\{
    \begin{array}{ll}
      0             & \text{if } j \geq n \\
      n - j         & \text{if } j < n
    \end{array}
  \right.
\end{align*}
Subtracting the two expressions yields $P_{A'',B}^{\Sigma}(i, j) = P_{A'A'',B}^{\Sigma}(i, j)$. It can be noticed that this conforms nicely with the computation carried out in case \circled{1}. Writing from the computation from case \circled{1},
\begin{align*}
  \circled{*} &= \min_{k \in \{0,\dots,n+m\}} \{ {IC'_{P_{A'}}}^{\Sigma}(i, k) + {IC''_{P_{A''}}}^{\Sigma}(k, j) \} \\
  &= \min\left\{
       \min_{k \in \{0,\dots,i\}}{P_{A'',B}^{\Sigma}(k, j)},
       \min_{k \in \{i + 1, n + m\}}\left\{ k - i + \left\{ % A row <= m'' in IC' is like: 0 ... 0 1 2 3 4 ...
        \begin{array}{ll}
          P_{A'',B}^{\Sigma}(k, j) & \text{if } k \leq n + m'' \\
          0                        & \text{if } k > n + m''
        \end{array}
       \right. \right\}
     \right\}.
\end{align*}
%
\begin{claim}
  \label{claim:bounding-sigma}
  The following bound is valid $\forall k \in \mathbb{N}: i + k \leq n + m''$:
  \[
    P_{A''}^{\Sigma}(i, j) \leq k + P_{A''}^{\Sigma}(i + k, j)
  \]
  \begin{proof}
    Follows almost directly from decreasing unit monotonicity of the $\Sigma$-operator used on permutation matrices in both coordinates (only the $i$ coordinate monotonicity used here).
  \end{proof}
\end{claim}
Using \cref{claim:bounding-sigma} at the boundary condition $i + k = n + m''$, it is found that $P_{A''}^{\Sigma}(i, j) \leq n + m'' - i + P_{A''}^{\Sigma}(n + m'', j) = n + m'' - i$. Hence, by first using decreasing unit monotonicity of the $\Sigma$-operator and \cref{claim:bounding-sigma} followed by the claim at the boundary condition, it is found that:
\[
  \circled{*} = \min\{ P_{A''}^{\Sigma}(i, j), \min_{k \in \{n+m'', \dots, n+m\}} \{ k - i \} \}
              = P_{A''}^{\Sigma}(i, j)
\]
Hence, the min-product computed in case \circled{3} also gives the correct result for case \circled{1}.

Case \circled{2} follows by almost the same arguments as case \circled{1}. First \cref{lemma:dist-H-relation} and the definition of the $H$-table is used on \cref{claim:dist-vertical-merge} yielding
\begin{align*}
  \DIST_{A'A'',B}[i, j] &= \DIST_{A',B}[i - m'', j - m''] \\
    &= H_{A',B}(i - m'', j - m'') +
      \begin{cases}
        (i - m'' - m') + (n - (j - m'')) & \text{if } i - m'' < m' \\
        n - (j - m'')                    & \text{if } i - m'' \geq m'
      \end{cases} \\
    &= n + m'' - j + H_{A',B}(i - m'', j - m'') +
      \begin{cases}
        i - m   & \text{if } i < m \\
        0       & \text{o/w}
      \end{cases} \\
    &= n - (i - m) - P_{A',B}^{\Sigma}(i - m'', j - m'') +
      \begin{cases}
        i - m   & \text{if } i < m \\
        0       & \text{o/w}
      \end{cases}
\end{align*}
Similarly writing out directly from the definition, it is found that
\begin{align*}
  \DIST_{A'A'',B}[i, j] &= H_{A'A'',B}(i, j) +
    \begin{cases}
      (i - m) + (n - j) & \text{if } i < m \\
      n - j             & \text{if } i \geq m
    \end{cases} \\
  &= n - (i - m) - P_{A'A'',B}^{\Sigma}(i, j) +
    \begin{cases}
      i - m & \text{if } i < m \\
      0     & \text{if } i \geq m
    \end{cases}
\end{align*}
This implies that $P_{A'A'',B}^{\Sigma}(i, j) = P_{A',B}^{\Sigma}(i - m'', j - m'')$. As will be seen, this is also what is computed by the min-multiplication in case \circled{3}. Writing out the result of the min-multiplication, it is found that
\begin{align*}
  \circled{*} &= \min_{k\in \{ 0, \dots, n + m \}} { {IC'_{P_{A'}}}^{\Sigma}(i, k) + {IC''_{P_{A''}}}^{\Sigma}(k, j) } \\
              &=  \begin{aligned} \min
                    &\left\{
                      \min_{k \in \{0, \dots, m''\}} { P_{A'}^{\Sigma}(k, n + m'') + I^{\Sigma}(0, j - (n + m'')) }, \right. \\
                    &\left.
                      \min_{k \in \{m', \dots, n + m''\}} {P_{A'}^{\Sigma}(i - m'', k - m'') + I^{\Sigma}(0, j - (n + m''))}, \right. \\
                    &\left.
                      \min_{k \in \{ n + m'', \dots, n + m \}} {P_{A'}^{\Sigma}(i - m'', k - m'') + I^{\Sigma}(k - (n + m''), j - (n + m''))}
                    \right\}
                  \end{aligned}
\end{align*}
Notice that the second case always dominates the first, hence the second case can safely be removed from the minimization. Furthermore, the third case can be spitted to obtain
\begin{align*}
  \circled{*} &=  \begin{aligned} \min
                    &\left\{
                      n + m'' + j - n - m'',
                      \min_{k \in \{n + m'', j\}} {j - k + P_{A'}^{\Sigma}(i - m'', k - m'')}, \right. \\
                    &\left.
                      \min_{k \in \{j, \dots, n + m\}} {P_{A'}^{\Sigma}(i - m'', k - m'')}
                    \right\}
                  \end{aligned} \\
              &= \begin{aligned} \min
                    &\left\{
                      j,\quad
                      \min_{k \in \{j, \dots, n + m\}} {j - k + P_{A'}^{\Sigma}(i - m'', k - m'')},\quad
                      P_{A'}^{\Sigma}(i - m'', j - m'')
                    \right\}
                  \end{aligned}
\end{align*}
\begin{claim}
  \label{claim:bounding-sigma-col}
  The following bound is valid $\forall a \in \mathbb{N}: j - a \geq 0$:
  \[
    P_{A'}^{\Sigma}(i - m'', j) \leq a + P_{A'}^{\Sigma}(i - m'', j - a)
  \]
  \begin{proof}
    Follows from unit monotonicity of the $\Sigma$-operator on permutation matrices.
  \end{proof}
\end{claim}
By applying \cref{claim:bounding-sigma-col} with $a = j - k$, it is found that
\[
  P_{A'}^{\Sigma}(i - m'', j - m'') \leq (j - k) + P_{A'}^{\Sigma}(i - m'', j - m'' - (j - k)) = (j - k) + P_{A'}^{\Sigma}(i - m'', k - m'').
\]
This eliminates the middle case of the minimization in $\circled{*}$. Moreover, \cref{claim:bounding-sigma-col} is used to find that
\[
  P_{A'}^{\Sigma}(i - m'', j - m'') \leq (j - m'') + P_{A'}^{\Sigma}(i - m'', 0) = j - m'' \leq j,
\]
which directly gives that $\circled{*} = P_{A'}^{\Sigma}(i - m'', j - m'')$. Therefore the min-multiplication can also be used to solve case \circled{2}. The above calculations establish the following important theorem.
\begin{theorem}[\refbook{p.-55}{Theorem 4.19}]
  Given two $\DIST$ tables $DIST_{A',B}$ and $\DIST_{A'',B}$ with representations $P_{A'}$ and $P_{A''}$ they can be merged into $\DIST_{A'A'',B}$ with representation $P_{A'A''}^{\Sigma}$ by setting
  \[
    P_{A'A''} = IC'_{P_{A'}} \boxdot IC''_{P_{A''}} .
  \]
\end{theorem}

 Therefore two $\DIST$ tables can be merged vertically by a single min-multiplication taking $O(x \log{x})$ time using the described in \cref{sec:algorithm:min-mult-two-unit-monge}.

\subsubsection{Horizontal merge}
The horizontal merge will be carried out by means of a vertical merge. In order to do this, it should be efficient to compute $\DIST_{B,A}$ from $\DIST_{A,B}$. It turns out that this indeed easy. In order to see this, first notice that
\begin{claim}[\refbook{p.-52}{Lemma 4.14, no proof}]
  \label{claim:H_string_switch_relation}
  For all $i, j \in \{0, \dots, n + m\}$ the following identity holds
  \[
    H_{A,B}(i, j) = (m - i) - (n - j) + H_{B,A}(n + m - i, n + m - j)
  \]
  \begin{proof}
    The proof is a case analysis of \cref{def:H-table}.
    \begin{description}
      \item[\circled{1}]: For $i \in \{0, \dots, m\}, j \in \{0, \dots, n\}$
        \begin{align*}
          H_{A,B}(i, j) &= (m - i) + lcs(\substr{A}{m - i}{m}, \substr{B}{0}{j}) \\
            &= (m - i) + lcs(\substr{B}{0}{j}, \substr{A}{m - i}{m}) \\
            &= (m - i) - (n - j) + lcs(B, \substr{A}{m - i}{m} ?^{n - j}) \\
            &= (m - i) - (n - j) + lcs(B, \substr{?^nA?^n}{n + m - i}{m + n + n - j}) \\
            &= (m - i) - (n - j) + H_{B,A}(n + m - i, n + m - j)
        \end{align*}

      \item[\circled{2}]: For $i \in \{m, \dots, m+n\}, j \in \{0, \dots, n\}$
        \begin{align*}
          H_{A,B}(i, j) &= lcs(A, \substr{B}{i - m}{j}) = lcs(\substr{B}{i - m}{j}, A) \\
            &= lcs(B, \substr{?^n A ?^n}{n - (i - m)}{n + m + (n - j)}) - (i - m) - (n - j) \\
            &= (m - i) - (n - j) + H_{B,A}(m + n - i, m + n - j)
        \end{align*}

      \item[\circled{3}]: For $i \in \{0, \dots, m\}, j \in \{n, \dots, n+m\}$
        \begin{align*}
          H_{A,B}(i, j) &= (m - i) - (n - j) + lcs(\substr{A}{m - i}{m - (j - n)}, B) \\
            &= (m - i) - (n - j) + lcs(B, \substr{?^n A ?^n}{m + n - i}{m + n - (j - n)}) \\
            &= (m - i) - (n - j) + H_{B,A}(m + n - i, m + n - j)
        \end{align*}

      \item[\circled{4}]: For $i \in \{m, \dots, m+n\}, j \in \{n, \dots, n+m\}$
        \begin{align*}
          H_{A,B}(i, j) &= (j - n+ lcs(\substr{A}{0}{m - (j - n)}, \substr{B}{i - m}{n})) \\
            &= (j - n) + lcs(\substr{B}{i - m}{n}, \substr{A}{0}{m + n - j}) \\
            &= (j - n) + lcs(B, \substr{?^n A ?^n}{n - (i - m)}{n + m + n - j}) - (i - m) \\
            &= (m - i) - (n - j) + H_{B,A}(n + m - i, m + n - j)
        \end{align*}
    \end{description}
  \end{proof}
\end{claim}
%
\begin{claim}
  \label{claim:sum_top_described_as_sum_bottom}
  For all $i, j \in \{0, \dots, n + m\}$ it is the case that
  \[
    \sum_{\substack{ \hat{i} \in \{i, \dots, n + m - 1\} \\ \hat{j} \in \{ 0, \dots, j - 1 \}}} P_{B,A}(m + n - \hat{i}, m + n - \hat{j})
      = j - i + P_{B,A}^{\Sigma}(m + n - i, m + n - j)
  \]
  \begin{proof}
    Since $P_{B,A}$ is a permutation matrix, it contains exactly one $1$ in every row/column. Therefore the sum over the top-right area, can be found by taking the sum over the entire matrix and subtract the left and bottom part, and then add the double counted overlap \todo{See figure ...}. This gives that
    \begin{align*}
      &\sum_{\substack{ \hat{i} \in \{i, \dots, n + m - 1\} \\ \hat{j} \in \{ 0, \dots, j - 1 \}}} P_{B,A}(m + n - \hat{i}, m + n - \hat{j}) \\
      = &(n + m) - (m + n - j) - ((m + n) - (m + n - i)) + P_{B,A}^{\Sigma}(m + n - i, m + n - j) \\
      = &j - i + P_{B,A}^{\Sigma}(m + n - i, m + n - j).
    \end{align*}
  \end{proof}
\end{claim}

\begin{lemma}[\refbook{p.-52}{Lemma 4.14, no proof}]
  \label{lemma:swap_strings}
  Assuming $P_{A,B}$ is the permutation matrix encoding $\DIST_{A,B}$, the $\DIST$-table $\DIST_{B,A}$ is encoded by the permutation matrix $P_{B,A}$ where,
  \[
    P_{B,A}(m + n - i, m + n - j) = P_{A,B}(i, j).
  \]
  Notice that the transformed permutation matrix $P_{B,A}$ can easily be computed in linear time by a sweep over the rows of $P_{A,B}$.
  \begin{proof}
    It is needed to show that the $H$-table corresponding to the permutation matrix $P_{B,A}$ conforms to \cref{claim:H_string_switch_relation}.
    \begin{align*}
      H_{A,B}(i, j) &= j - (i - m) - P_{A,B}^{\Sigma}(i, j) \\
        &= j - (i - m) - \sum_{\substack{ \hat{i} \in \{i,\dots,n+m - 1\} \\ \hat{j} \in \{0, \dots, j - 1\}}} P_{A,B}(\hat{i}, \hat{j}) \\
        &= j - (i - m) - \sum_{\substack{ \hat{i} \in \{i, \dots, n + m - 1\} \\ \hat{j} \in \{ 0, \dots, j - 1 \}}} P_{B,A}(m + n - \hat{i}, m + n - \hat{j}) \\
        &\overset{\text{\cref{claim:sum_top_described_as_sum_bottom}}}{=} j - (i - m) - (j - i + P_{B,A}^{\Sigma}(m + n - i, m + n - j)) \\
        &= m - P_{B,A}^{\Sigma}(m + n - i, m + n - j) \\
        &= m - \left( (n + m - j) - ((n + m - i) - n) - H_{B,A}(n + m - i, n + m - j) \right) \\
        &= (m - i) - (n - j) + H_{B,A}(n + m - i, n + m - j)
    \end{align*}
    This matches the statement of \cref{claim:H_string_switch_relation}, which completes the proof.
  \end{proof}
\end{lemma}
%
Following \cref{lemma:swap_strings} a horizontal merge can be carried out by first swapping the strings, then performing a vertical merge, and then swapping the strings of the merged $\DIST$ table again. Since swapping the strings runs in time $O(x)$, the total time of a horizontal merge is bounded by the time of the vertical merging. Hence, the total time of either of the merge routines is bounded by $O(x\log{x})$.

\subsection{Applying a $\DIST$ table}
\label{sec:applying-a-dist-table}
When filling out a block in the grid with input $I$, the output is given by \cref{eqn:dist-application} which by \cref{lemma:dist-H-relation} can be re-written as
\begin{align*}
  O[j] &= \max_i \left\{ I[i] + \DIST[i, j] \right\} \\
    &= \max_i \left\{ I[i] + H(i, j) + \left\{
      \begin{array}{ll}
        i - m             & \text{if } i < m, j < n \\
        0                 & \text{if } i \geq m, j < n \\
        (i - m) + (n - j) & \text{if } i < m, j \geq n \\
        n - j             & \text{if } i \geq m, j \geq n
      \end{array}
    \right. \right\} \\
    &= \max_i\left\{ \left( I[i] + \left\{
      \begin{array}{ll}
        i - m     & \text{if } i < m \\
        0         & \text{o/w.}
      \end{array}
    \right. \right) + H(i, j) \right\} + \left\{
      \begin{array}{ll}
        n - j     & \text{if } j \geq n \\
        0         & \text{o/w.}
      \end{array}
    \right. .
\end{align*}
Hence, the output can be evaluated by max-multiplying the $H$ table with a slight perturbation of the input array, followed by an addition for every resulting output. \cite{Gawrychowski:2012:FAC:2422024.2422048} gave a linear time algorithm for max-multiplying an arbitrary vector with a $H$-table which is described in \cref{sec:algorithm:max-mult-H-table-with-vector}. This gives a linear time algorithm for applying a $\DIST$ table.

\section{Efficient simple unit-Monge operations}
The two basic building blocks for merging and applying DIST tables are the following two algorithms described in this section. They both rely on properties of simple unit-Monge matrices in order to gain faster computation that otherwise possible.

\subsection{Min-multiplication of two simple unit-Monge matrices}
\label{sec:algorithm:min-mult-two-unit-monge}
\todo{ref book}
This section will describe how the minimum distance product of two simple unit-Monge matrices of size $n$ can be calculated in $O(n\log{n})$ time. The algorithm is the main result of \cite{Tiskin:2010:FDM:1873601.1873704}. The explanation follows the one in \cite{Tiskin:2010:FDM:1873601.1873704}, however there seems to be an error (or unexplained step) in the end of the algorithm, which is fixed in this description.

It is assumed from now on that the input matrices are given by their permutation matrices $P_A$ and $P_B$ both of size $n$. The result of the algorithm should be a permutation matrix $P_C$ such that $P_A \boxdot P_B = P_C$. The algorithm is a divide and conquer algorithm on the input permutation matrices $P_A$ and $P_B$.

The computation is trivial in the base case $n = 1$ since only one permutation matrix of this size exists. In the actual implementation the base case size was chosen to be $k \geq 1$. The reason for choosing a bigger base case is that the constant involved in the algorithm makes it slower than direct evaluation for small input sizes. In practice it was found that $k \approx 20$ was the best performing configuration \todo{link to some experiments?}.

For the recursive step the two input permutation matrices are first divided into four of size $\frac{n}{2}$ and two recursive calls are then performed. More specifically the subproblems are defined as
\begin{align*}
  P_{A,lo} = P_A\left[*, 0 : \frac{n}{2}\right], \quad &P_{B,lo} = P_B\left[0 : \frac{n}{2}, *\right] \\
  P_{A,hi} = P_A\left[*, \frac{n}{2} : n\right], \quad &P_{B,hi} = P_B\left[\frac{n}{2} : n, *\right].
\end{align*}
\todo{graphical illustration} Notice that all of these sub-problems are rectangular matrices of size $n \times \frac{n}{2}$ or $\frac{n}{2} \times n$. However, half of the rows (or columns) contains all zero entries, and they can therefore be virtually removed by doing an appropriate index mapping when doing recursive calls. The results of the recursive calls are denoted by
\[
  P_{C,lo} = P_{A,lo} \boxdot P_{B,lo}, \quad P_{C,hi} = P_{A,hi} \boxdot P_{B,hi}.
\]
As a result from the recursive calls these matrices are of size $\frac{n}{2}$. However, by doing the inverse index remapping from the above step, they can be considered as matrices of size $n$ and with the property that $P_{C,lo} + P_{C,hi}$ is a permutation matrix, since they come from disjoint ranges. From this point on $P_{C,lo}$ and $P_{C,hi}$ will be considered as matrices of size $n$. Notice that when the $\Sigma$-operator is applied to the sub-problem matrices, all the rows (columns) that contain all zeros in the sub-problem matrix will be identical to the previous row (column) in the $\Sigma$'ed version. Therefore all the sub-problem matrices where $\Sigma$ is applied will be indexed as $n \times \frac{n}{2}$ or $\frac{n}{2} \times n$ matrices as this eases index notation.

In the following calculations, let $i, j \in \{0, \dots, n\}$. First notice that the desired result can be written as
\begin{align*}
  P_C^{\Sigma}(i, j) &= \min_{k \in \{0, \dots n\}} \left( P_A^{\Sigma}(i, k) + P_B^{\Sigma}(k, j) \right) \\
    &= \min\left( \min_{k \in \{0, \dots, \frac{n}{2}\}} {P_A^{\Sigma}(i, k) + P_B^{\Sigma}(k, j)}, \min_{k \in \{\frac{n}{2}, \dots, n\}} {P_A^{\Sigma}(i, k) + P_B^{\Sigma}(k, j)} \right).
\end{align*}
The goal is to reduce the two sums inside the minimization to a constant number of table lookups in the results from the recursive calls. To do this notice that
\begin{align*}
  P_{C,hi}^{\Sigma}(0, j) &= \min_{k \in \{0, \dots, \frac{n}{2}\}} \left( P_{A,hi}^{\Sigma}(0, k) + P_{B,hi}^{\Sigma}(k, j) \right) \\
    &= \min_{k \in \{0, \dots, \frac{n}{2}\}} \left( k + P_{B,hi}^{\Sigma}(k, j) \right) = P_{B,hi}^{\Sigma}(0, j)
  \\ \\
  P_{C,lo}^{\Sigma}(i, n) &= \min_{k \in \{0, \dots, \frac{n}{2}\}} \left( P_{A,lo}^{\Sigma}(i, k) + P_{B,lo}^{\Sigma}(k, n) \right) \\
    &= \min_{k \in \{0, \dots, \frac{n}{2}\}} \left( P_{A,lo}^{\Sigma}(i, k) + \frac{n}{2} - k \right)
    = P_{A,lo}^{\Sigma}(i, \frac{n}{2})
\end{align*}
where the last equality in both statements follows since $P_{B,hi}$ and $P_{A,lo}$ are sub-permutation\footnote{A sub-permutation matrix is a permutation matrix but where some of the $1$ entries may be omitted.} matrices and hence can decrease by at most one every time $k$ is increased. These two observations can be used to find that
\begin{align*}
  \min_{k \in \{0, \dots, \frac{n}{2}\}} \left( P_A^{\Sigma}(i, k) + P_B^{\Sigma}(k, j) \right)
    &= \min_{k \in \{0, \dots, \frac{n}{2}\}} \left( P_{A,lo}^{\Sigma}(i, k) + P_{B,lo}^{\Sigma}(k, j) + P_{B,hi}^{\Sigma}(0, j) \right) \\
    &= P_{C,lo}^{\Sigma}(i, j) + P_{B,hi}^{\Sigma}(0, j)
    = P_{C,lo}^{\Sigma}(i, j) + P_{C,hi}^{\Sigma}(0, j)
  \\ \\
  \min_{k \in \{\frac{n}{2}, \dots, n\}} \left( P_A^{\Sigma}(i, k) + P_B^{\Sigma}(k, j) \right)
    &= \min_{k \in \{0, \dots, \frac{n}{2}\}} \left( P_{A,hi}^{\Sigma}(i, k) + P_{A,lo}^{\Sigma}(i, \frac{n}{2}) + P_{B,hi}^{\Sigma}(k, j) \right) \\
    &= P_{C,hi}^{\Sigma}(i, j) + P_{A,lo}^{\Sigma}(i, \frac{n}{2})
    = P_{C,hi}^{\Sigma}(i, j) + P_{C,lo}^{\Sigma}(i, n),
\end{align*}
which give the following correspondence between the result from the recursive calls and the result of the current call
\[
  P_C^{\Sigma}(i, j) = \min\left( P_{C,lo}^{\Sigma}(i, j) + P_{C,hi}^{\Sigma}(0, j),
                                  P_{C,hi}^{\Sigma}(i, j) + P_{C,lo}^{\Sigma}(i, n)
                            \right).
\]
However, there is not enough time to loop through all combinations of $i,j$ since this would take $O(n^2)$ time. Instead three conditions will be established that gives the positions of the $1$'s in $P_C$. In order to do this, the difference between the two arguments in the minimization is investigated. Define:
\begin{align*}
  \delta(i, j) :&= \left( P_{C,lo}^{\Sigma}(i, j) + P_{C,hi}^{\Sigma}(0, j) \right) - \left( P_{C,hi}^{\Sigma}(i, j) + P_{C,lo}^{\Sigma}(i, n) \right) \\
    &= \left( P_{C,hi}^{\Sigma}(0, j) - P_{C,hi}^{\Sigma}(i, j) \right) - \left( P_{C,lo}^{\Sigma}(i, n) - P_{C,lo}^{\Sigma}(i, j) \right) \\
    &= \left( \sum_{\substack{\hat{i} \in \{0, \dots, n - 1\} \\ \hat{j} \in \{0, \dots, j - 1\}}} P_{C,hi}(\hat{i}, \hat{j}) - \sum_{\substack{\hat{i} \in \{i, \dots, n - 1\} \\ \hat{j} \in \{0, \dots, j - 1\}}} P_{C,hi}(\hat{i}, \hat{j}) \right) \\ &\quad\quad- \left( \sum_{\substack{\hat{i} \in \{i, \dots, n - 1\} \\ \hat{j} \in \{0, \dots, n - 1\}}} P_{C,lo}(\hat{i}, \hat{j}) - \sum_{\substack{\hat{i} \in \{i, \dots, n - 1\} \\ \hat{j} \in \{0, \dots, j - 1\}}} P_{C,lo}(\hat{i}, \hat{j}) \right) \\
    &= \sum_{\substack{\hat{i} \in \{0, \dots, i - 1\} \\ \hat{j} \in \{0, \dots, j - 1\}}} P_{C,hi}(\hat{i}, \hat{j}) - \sum_{\substack{\hat{i} \in \{i, \dots, n - 1\} \\ \hat{j} \in \{j, \dots, n - 1\}}} P_{C,lo}(\hat{i}, \hat{j})
\end{align*}
Since $P_{C,lo} + P_{C,hi}$ is a permutation matrix, $\delta$ is unit-monotone increasing in both of its arguments. The sign of $\delta(i, i)$ plays a vital role in determining the position of the $1$'s in $P_C$. The following three exhaustive mutual exclusive cases are considered:
\begin{description}
  \item[$\delta(i + 1, j + 1) \leq 0$]: From the monotonicity of $\delta$ it is known that $\delta(k, l) \leq 0$ for all $k \leq i + 1$ and $j \leq j + 1$. Therefore the definition of the $\delta$-function gives that for all such $k$ and $l$'s it is the case that
  \[
    P_C^{\Sigma}(k, l) = P_{C,lo}^{\Sigma}(k, l) + P_{C,hi}^{\Sigma}(0, l).
  \]
  By using the above identity, \cref{claim:sigma-box-identity} and unfolding definitions it is found that
  \begin{align*}
    P_C(i, j) &= (P_C^{\Sigma})^{\Box}(i, j) = P_C^{\Sigma}(i, j + 1) - P_C^{\Sigma}(i, j) - P_C^{\Sigma}(i + 1, j + 1) + P_C^{\Sigma}(i + 1, j) \\
    &= (P_{C,lo}^{\Sigma}(i, j + 1) + P_{C,hi}^{\Sigma}(0, j + 1)) - (P_{C,lo}^{\Sigma}(i, j) + P_{C,hi}^{\Sigma}(0, j)) \\&\quad\quad - (P_{C,lo}^{\Sigma}(i + 1, j + 1) + P_{C,hi}^{\Sigma}(0, j + 1)) + (P_{C,lo}^{\Sigma}(i + 1, j) + P_{C,hi}^{\Sigma}(0, j)) \\
    &= P_{C,lo}^{\Sigma}(i, j + 1) - P_{C,lo}^{\Sigma}(i, j) - P_{C,lo}^{\Sigma}(i + 1, j + 1) + P_{C,lo}^{\Sigma}(i + 1, j) \\
    &= (P_{C,lo}^{\Sigma})^{\Box}(i, j) = P_{C,lo}(i, j).
  \end{align*}

  \item[$\delta(i, j) \geq 0$]: This case is very similar to the first. By monotonicity of $\delta$ it is known that $\delta(k, l) \geq 0$ for all $k \geq i, l \geq l$. For all such $k$ and $l$'s the definition of $\delta$ now gives that
  \[
    P_C^{\Sigma}(i, j) = P_{C,hi}^{\Sigma}(i, j) + P_{C,lo}^{\Sigma}(i, n).
  \]
  Like before \cref{claim:sigma-box-identity} is used to find that
  \begin{align*}
    P_C(i, j) &= (P_C^{\Sigma})^{\Box}(i, j) = P_C^{\Sigma}(i, j + 1) - P_C^{\Sigma}(i, j) - P_C^{\Sigma}(i + 1, j + 1) + P_C^{\Sigma}(i + 1, j) \\
    &= (P_{C,hi}^{\Sigma}(i, j + 1) + P_{C,lo}^{\Sigma}(i, n)) - (P_{C,hi}^{\Sigma}(i, j) + P_{C,lo}^{\Sigma}(i, n)) \\&\quad\quad - (P_{C,hi}^{\Sigma}(i + 1, j + 1) + P_{C,hi}^{\Sigma}(i + 1, n)) + (P_{C,hi}^{\Sigma}(i + 1, j) + P_{C,lo}^{\Sigma}(i + 1, n)) \\
    &= P_{C,hi}^{\Sigma}(i, j + 1) - P_{C,hi}^{\Sigma}(i, j) - P_{C,hi}^{\Sigma}(i + 1, j + 1) + P_{C,hi}^{\Sigma}(i + 1, j) \\
    &= (P_{C,hi}^{\Sigma})^{\Box}(i, j) = P_{C,hi}(i, j).
  \end{align*}

  \item[$\delta(i + 1, j + 1) > 0, \delta(i, j) < 0$]: By unit monotonicity of $\delta$ it must be the case that
  \[
    \delta(i + 1, j) = \delta(i, j + 1) = 0.
  \]
  From the definition of $\delta$, this implies that
  \begin{align*}
    P_C^{\Sigma}(i, j + 1) &= P_{C,lo}^{\Sigma}(i, j + 1) + P_{C,hi}^{\Sigma}(0, j + 1) = P_{C,hi}^{\Sigma}(i, j + 1) + P_{C,lo}^{\Sigma}(i, n) \\
    P_C^{\Sigma}(i + 1, j) &= P_{C,lo}^{\Sigma}(i + 1, j) + P_{C,hi}^{\Sigma}(0, j) = P_{C,hi}^{\Sigma}(i + 1, j) + P_{C,lo}^{\Sigma}(i + 1, n).
  \end{align*}
  Also the assumptions on $\delta$ gives that
  \begin{align*}
    P_C^\Sigma(i, j) &= P_{C,lo}^{\Sigma}(i, j) + P_{C,hi}^{\Sigma}(0, j) < P_{C,hi}^{\Sigma}(i, j) + P_{C,lo}^{\Sigma}(i, n) \\
    P_C^{\Sigma}(i + 1, j + 1) &= P_{C,hi}^{\Sigma}(i + 1, j + 1) + P_{C,lo}^{\Sigma}(i + 1, n) < P_{C,lo}^{\Sigma}(i + 1, j + 1) + P_{C,hi}(0, j + 1).
  \end{align*}
  Using these facts and \cref{claim:sigma-box-identity} it is found that
  \begin{align*}
    P_C(i, j) &= P_C^{\Sigma}(i, j + 1) - P_C^{\Sigma}(i, j) - P_C^{\Sigma}(i + 1, j + 1) + P_C^{\Sigma}(i + 1, j) \\
    &> (P_{C,lo}^{\Sigma}(i, j + 1) + P_{C,hi}^{\Sigma}(0, j + 1)) - (P_{C,lo}^{\Sigma}(i, j) + P_{C,hi}^{\Sigma}(0, j)) \\&\quad\quad - (P_{C,lo}^{\Sigma}(i + 1, j + 1) + P_{C,hi}(0, j + 1)) + (P_{C,lo}^{\Sigma}(i + 1, j) + P_{C,hi}^{\Sigma}(0, j)) \\
    &= P_{C,lo}^{\Sigma}(i, j + 1) - P_{C,lo}^{\Sigma}(i, j) - P_{C,lo}^{\Sigma}(i + 1, j + 1) + P_{C,lo}^{\Sigma}(i + 1, j) \\
    &= P_{C,lo}(i, j)
  \end{align*}
  Since both $P_{C,lo}$ and $P_C$ by \cref{claim:unit-monge-min-prod-closed} are permutation matrices and the strict equality $P_C(i, j) > P_{C,lo}(i, j)$ is established, it must be the case that $P_C(i, j) = 1$.
\end{description}
Summarizing the above cases, $P_C(i, j)$ should contain a $1$ iff one of the following mutual exclusive conditions is satisfied
\begin{align}
  P_{C,lo}(i, j) = 1 &\text{ and } \delta(i + 1, j + 1) \leq 0 \\
  P_{C,hi}(i, j) = 1 &\text{ and } \delta(i, j) \geq 0 \\
  \delta(i + 1, j + 1) > 0 &\text{ and } \delta(i, j) < 0.
\end{align}
The idea is to make it efficiently to check when one of the condition is satisfied. In order to do this, is is needed to be able to efficiently check if the $\delta$ function contains a number $< 0$ or $> 0$. If for a given $i, j$ this can be done in constant time, all the above $\delta$-constraints can be efficiently checked.

In order to facilitate this, the $\delta$ function is viewed as a $n \times n$ matrix. From unit monotonicity the entries $< 0$ must be located in the top left part, and the entries $> 0$ must be in the bottom right path as illustrated by \todo{insert illustration}. Basically the idea now is to trace two separating paths, separating the entries $< 0$ and the ones $> 0$ from the rest. The separating bath for the entries $< 0$ are called $r_{hi}$ and the other is called $r_{low}$.

Each of these paths are of length exactly $2n$, and the idea is to represent a path, say $r_{hi}$ as an array $r_{hi}$ of length $2n$ such that an entry on the path is given by
\[
  \left( \frac{r_{hi}(d) - d}{2}, \frac{r_{hi}(d) + d}{2} \right)
\]
where $d \in \{-n, \dots, n\}$. That is, if $r_{hi}(-n) := n$, then the starting position becomes $(n, 0)$ which is the desired starting position. Let $d \in \{-n, \dots, n - 1\}$, then if the path should go upward set $r_{hi}(d + 1) = r_{hi}(d) - 1$ because
\begin{align*}
  \frac{r_{hi}(d + 1) - (d + 1)}{2} &= \frac{r_{hi}(d) - 1 - (d + 1)}{2} = \frac{r_{hi}(d) - d}{2} - 1 \quad \text{and} \\
  \frac{r_{hi}(d + 1) + (d + 1)}{2} &= \frac{r_{hi}(d) - 1 + (d + 1)}{2} = \frac{r_{hi}(d) - d}{2}.
\end{align*}
Similarly if the path should go right, then set $r_{hi}(d + 1) = r_{hi}(d) + 1$ since
\begin{align*}
  \frac{r_{hi}(d + 1) - (d + 1)}{2} &= \frac{r_{hi}(d) + 1 - (d + 1)}{2} = \frac{r_{hi}(d) - d}{2} \quad \text{and} \\
  \frac{r_{hi}(d + 1) + (d + 1)}{2} &= \frac{r_{hi}(d) + 1 + (d + 1)}{2} = \frac{r_{hi}(d) - d}{2} + 1.
\end{align*}
Exactly the same encoding of the path is used for $r_{low}$. In order to actually trace the separating paths, it is required that a $\delta$ entry called the \textit{witness}, adjacent to the current position in the path can be queried in constant time. For the high path, the witness is the entry just above the current entry and for the low path it is the one immediately to the right of the current position. In order to compute the witness in constant time, the approach taken in \cref{lemma:simple-unit-monge-query-next} can be almost directly reused as the $\delta$ function is a combination of sums over two permutation matrices.

Since the witness can be computed in constant time, it takes $O(n)$ times to build the two separating paths. Now the $\delta$ checks can be simplified as follows:
\begin{description}
  \item[$\delta(i + 1, j + 1) \leq 0$]: This case happens iff $(i + 1, j + 1)$ is above or on the $r_{lo}$ path. Let $d = (n - (i + 1) + (j + 1)) - n = j - i$ be the number of steps that is required for the path to meet the $(i + 1, j + 1)$ entry. Then $(i + 1, j + 1)$ is above or on $r_{lo}$ iff
  \begin{align*}
         &\frac{r_{lo}(j - i) + j - i}{2} = \frac{r_{lo}(d) + d}{2} \geq j + 1 \\
    \iff &r_{lo}(j - i) + j - i \geq 2(j + 1) \\
    \iff &r_{lo}(j - i) \geq i + j + 2
  \end{align*}

  \item[$\delta(i, j) \geq 0$]: This case happens iff $(i, j)$ is below or on $r_{hi}$. Again define $d = (n - i) + j - n = j - i$ to be the number of steps required to reach entry $(i, j)$. Then $(i, j)$ is below or on $r_{hi}$ iff
  \begin{align*}
         &\frac{r_{hi}(j - i) + j - i}{2} = \frac{r_{hi}(d) + d}{2} \leq j \\
    \iff &r_{hi}(j - i) + j - i \leq 2j \\
    \iff &r_{hi}(j - i) \leq i + j
  \end{align*}

  \item[$\delta(i + 1, j + 1) > 0, \delta(i, j) < 0$]: This happens iff $(i + 1, j + 1)$ is under $r_{lo}$ and $(i, j)$ is above $r_{hi}$. Notice that the number of steps required to reach any of these entries is $d = j - i$. Therefore using the same reasoning as in the previous two cases the condition is satisfied iff both of the following statements are satisfied
  \begin{itemize}
  \item $r_{lo}(j - i) < j + i + 2$
  \item $r_{hi}(j - i) > j + i$
  \end{itemize}
  Or summarized into a single line iff $r_{lo}(j - i) - 2 < j + i < r_{hi}(j - i)$.
  By the unit monotonicity of $\delta$, $r_{lo}$ will be located below $r_{hi}$. For this reason, it will be the case that $r_{lo}(d') \geq r_{hi}(d')$ for all $d' \in \{-n, \dots, n\}$, which squeezes the possible values of $r_{hi}(j - i) \in \{r_{lo}(j - i), r_{lo}(j - i) + 1\}$. However, by construction it must always be the case that $r(d') + d'$ is even, for both $r_{lo}$ and $r_{hi}$. Hence it follows that the condition is satisfied iff $r_{hi}(j - i) = r_{lo}(j - i)$.
\end{description}%
%
The check if $P_C(i, j)$ should contain a $1$ can now be formulated as
\begin{align}
  &P_{C,lo}(i, j) = 1 \text{ and } r_{lo}(j - i) \geq i + j + 2 \\
  &P_{C,hi}(i, j) = 1 \text{ and } r_{hi}(j - i) \leq i + j \\
  &r_{lo}(j - i) = r_{hi}(j - i).
\end{align}
The first two cases can be checked by looping through all the $1$'s in both $P_{C,lo}$ and $P_{C,hi}$ and then for each of them check the corresponding path condition. Each check takes constant time and there are in total $n$ $1$'s in the two matrices. The $1$'s from the first two cases can thereby be computed in time $O(n)$.

For the last case the idea is to loop through all $d \in \{-n, \dots, n\}$ and then check if $r_{lo}(d) = r_{hi}(d)$. If this is the case, then the position corresponding to that path position should be set to $1$. The corresponding path position is by definition given to be
\[
  (i, j) = \left( \frac{r(d) - d}{2}, \frac{r(d) + d}{2} \right).
\]
Therefore the $1$'s from the last case can also be computed in time $O(n)$.

The total time for the algorithm then becomes
\[
  T(n) = O(n) + 2 T\left(\frac{n}{2}\right)
\]
which is well known and solves to $O(n \log{n})$.

\subsection{Max-multiplication of $H$-table with vector}
\label{sec:algorithm:max-mult-H-table-with-vector}
In order to apply the DIST tables to fill out the grid, it is required to be able to compute the maximum distance product between an arbitrary vector and a $H$-table. This is a description of the algorithm given in \cite[Lemma 2, p. 234]{Gawrychowski:2012:FAC:2422024.2422048}.

Let $v$ denote the input vector of size $n$ and assume we are given a $n \times n$ $H$-table $H$. The algorithm incrementally computes the maximum distance product where each iteration takes $O(1)$ time, leading to a $O(n)$ time algorithm. The output of the algorithm should be
\[
  u(j) = \max_i v(i) + H(i, j) = \max_i v(i) + j - (i - m) - P^{\Sigma}(i, j)
\]
where $P$ is the permutation matrix representation of the $H$-table. By defining $v'(i) = v(i) - i$ and $u'(j) = u(j) - j$ the maximization can be simplified to
\[
  u'(j) = \max_i v'(i) - P^{\Sigma}(i, j).
\]
In the implementation it is not necessary to copy the input, but the slightly modified input and output vectors can be "simulated" directly from the original input. However, the redefinition makes the description of the algorithm a bit more clear, and makes it follow \cite[Lemma 2, p. 234]{Gawrychowski:2012:FAC:2422024.2422048}.

Define $t_j(i) := v'(i) - P^{\Sigma}(i, j)$. That is, for a fixed $j$ it contains all the entries from where the maximum entry needs to be found. Assume there exist two indices $i < i'$ such that $t_j(i) \leq t(i')$, then because $P$ is a permutation matrix and by the definition of the $\Sigma$-operator, there are two possibilities:
\begin{description}
  \item[$P^{\Sigma}(i', j) = P^{\Sigma}(i', j + 1) - 1$]: In this case it follows by the definition of the $\Sigma$-operator and that $P$ is a permutation matrix that $P^{\Sigma}(i, j) = P^{\Sigma}(i, j + 1) - 1$.  This can be used to find that
  \begin{align*}
    &t_{j + 1}(i) = v'(i) - P^{\Sigma}(i, j + 1) = v'(i) - P^{\Sigma}(i, j) - 1
      = t_{j}(i) - 1 \\
      \leq\quad &t_j(i') - 1 = v'(i') - P^{\Sigma}(i', j) - 1
      = v'(i') - (P^{\Sigma}(i', j + 1) - 1) - 1 \\
      &= v'(i') - P^{\Sigma}(i', j + 1) = t_{j + 1}(i').
  \end{align*}

  \item[$P^{\Sigma}(i', j) = P^{\Sigma}(i', j + 1)$]: A similar calculation can be used in this case. First notice that $P^{\Sigma}(i, j) \leq P^{\Sigma}(i, j + 1)$ which follows since $P$ is non-negative. It is found that
  \begin{align*}
    &t_{j + 1}(i) = v'(i) - P^{\Sigma}(i, j + 1) \leq v'(i) - P^{\Sigma}(i, j) = t_{j}(i) \\
      \leq\quad &t_j(i') = v'(i') - P^{\Sigma}(i', j) = v'(i') - P^{\Sigma}(i', j + 1) = t_{j + 1}(i').
  \end{align*}
\end{description}
%
Therefore, once such two indices $i,i'$ are found, it is in all successive iterations not necessary to consider the entry at position $i$ as $i'$ will always be at least as good.

The idea now is to store a list of candidates that can attain the maximum possible values, and then weed out this list by using the above observation. Therefore a list of candidate indices $i_1 < i_2 < \dots < i_l$ is maintained such that $t(i_1) > t(i_2) > \dots > t(i_l)$, where $t$ denotes the $t$-values for the current value of $j$. This also means that the result of a given iteration is always given by $t(i_1)$.

The first such index list is easy to compute by sweeping from right to left and then pick out the sequence of strictly increasing $t$-values.

Notice that by the definition of the $\Sigma$-operator, all entries $t(1), \dots, t(k)$ should be decreased by $1$, where $k$ is the row number of the $1$ entry in row $j$ in $P$. In order to be able to do this in constant time, the selected candidates of $t$ are represented by their relative difference. A bit more details on this is given in \cite[Lemma 2, p. 234]{Gawrychowski:2012:FAC:2422024.2422048}.

Doing this, a prefix of the entries can be decreased by decreasing the first entry, and then find the leftmost entry $> k$, and increase this difference by $1$. If some difference reaches $0$ it is removed from the list, as this can never again be a candidate from the condition above.

The only thing that remains is to locate the leftmost candidate $> k$. This can be done using whats called an interval union find data structure. Such a data structure is initialized with an interval of length $n$. In the beginning there are $n$ disjoint, contiguous and evenly distributed intervals. The data structure support the operation to merge two adjacent intervals, and to query the last point contained in some interval.

The interval union find data structure is used as follows: In the beginning all the entries in the $t$-array have their own interval. As more and more candidates are removed from the list, their intervals are joined. Finding the leftmost candidate $> k$ can then be done by issuing a single interval union find query, taking $O(1)$ time \cite{Itai06lineartime}.

\subsubsection{Note on Interval Union Find}
\label{sec:algorithm:interval-union-find}
The interval union find data-structure implemented is the one of \cite{Itai06lineartime}. The algorithm works by blocking the interval into blocks fitting a machine word of size $\Theta(\log{n})$. Operations inside such a block can be supported very efficiently using simple bit operations. Operations spanning blocks are carried out by using an ordinary union find data structure. \cite{Itai06lineartime} shows that this leads to a data-structure using amortized constant time operations.

 Note however that the amortized constant time relies on blocking the input into blocks of size $\Theta(\log{n})$. In the actual implementation, it has been chosen to always use block sizes of machine word $64$ bits, which is a constant. This choice will always perform superior in practice due to practical limit on the input size, but the theoretical running time of the algorithm becomes that of the ordinary union find data structure, which is the inverse Ackermann function. However, for all practical purposes this is also is a constant. Therefore it is not to be expected that the choice of the constant block size will result in weird looking results in the graphs in the following benchmarks.

\clearpage
\section{Benchmarks}
\label{sec:algorithm:benchmarks}
This section will benchmark the compression based edit distance algorithm. Both with respect to the theoretical running time and also by comparing it to the performance of the simple algorithm. First it will be verified that the distance product algorithms performs as expected, and then the combined algorithm for computing the edit distance of compressed strings will be evaluated.

\subsection{Distance products}
In this section the distance product operations on unit-Monge matrices will be benchmarked and optimized for later use.

\subsubsection{Minimum distance product}
In order to get the best possible performance for the minimum distance product computation, the size of the base case of the algorithm described in \cref{sec:algorithm:min-mult-two-unit-monge} should be determined. Notice that when benchmarking the algorithm, the same amount of work should be done no matter the input. Therefore the running time should not depend on the content of the input, but only on its size. Therefore, all the benchmarks of the minimum distance product algorithms have been done with random permutation matrices of a specific size.

\Cref{fig:benchmark:min-distance:bc} plots the running time normalized by $n\log{n}$ of the algorithm when varying the base case size. The general form of the plot is as expected: A small base case size leads to a larger running time, and a too large base case size will also yield a larger running time. The best compromise seems to be around base case size $k \approx 20$. For all the following measurements in this report a base case size of $k = 20$ has been used.

Another thing to notice about \cref{fig:benchmark:min-distance:bc} is that it looks like, for a fixed base case size, that the normalized running time converges to a constant. This suggests that the theoretical running time is also satisfied in practice.

\begin{figure}[h!]
  \centering
  \includegraphics[width=10cm]{distance-mult/min-dist-mult-bc}
  \caption{Normalized running time of the minimum distance product algorithm for varying base case sizes.}
  \label{fig:benchmark:min-distance:bc}
\end{figure}

Notice that the normalized running times for the base case size of $50$ experience periodic peaks. These are due to the actual base case size when running the algorithm. In every step the algorithm splits the input size by $2$ until the input size gets smaller than the base case size. Therefore assuming that the input size is always divisible by $2$, the number of times the input is split becomes $x = \lceil \log{\frac{N}{k}} \rceil$ where $k$ is the base case size. That is, the actual instance size when the base case is hit becomes $\frac{N}{2^x}$. As can be seen on \cref{fig:benchmark:min-distance:bc50} this correlates very nicely with the running time of the algorithm. The reason these jumps are not seen for smaller base case sizes is that the simple unfolding approach is fast for small base case sizes.

\begin{figure}[h!]
  \centering
  \includegraphics[width=10cm]{distance-mult/min-dist-mult-bc50}
  \caption{Examination of the jumps occurring when the base case size is 50.}
  \label{fig:benchmark:min-distance:bc50}
\end{figure}

\subsubsection{Maximum distance product}
This section will examine if the maximum distance product algorithm runs within the theoretical running time, and its running time will be compared to a simple unfolding algorithm.

This algorithm is a bit more input sensitive than the one for computing the minimum distance product, since a list of candidates is maintained, where the length of this may influence the running time of the instance. It has been chosen to verify the running time using randomly generated input. This choice is not directly related to the context where it will be used later, however the input should be fine to test if the theoretical running time is satisfied.
\todo{Consider input data.}

\Cref{fig:benchmark:max-distance} plots the normalized running time of both the naive $O(n^2)$ algorithm, and the efficient $O(n)$ algorithm of \cite{Gawrychowski:2012:FAC:2422024.2422048}. It can be seen that efficient algorithm is always superior to the naive, hence the efficient algorithm is used for all input sizes.

Moreover \cref{fig:benchmark:max-distance} indicates that the normalized running time approaches a constant as the input size grows. Therefore it seems like the implementation satisfies the theoretical running time bound.

\begin{figure}[h!]
  \centering
  \includegraphics[width=10cm]{distance-mult/max-dist-mult}
  \caption{Running time of the efficient and naive algorithms for computing the maximum distance product of a permutation matrix and a vector.}
  \label{fig:benchmark:max-distance}
\end{figure}

\subsection{Combined algorithm}
This section will present and discuss measurements on the implementation following together with this thesis. First it will be examined that the two main parts satisfies their theoretical running time, and the time distribution between the two steps will be examined.

In the end it will be checked that all the steps combined into the complete algorithm satisfies the theoretical running time, and the running time will be compared to the running time of the simple algorithm on the same inputs.

\subsubsection{DIST Repository and grid computation}
\todo{Verification of theoretical running time for DIST/grid.}

The increase in the normalized running time for building the DIST repository in the interval $[2^3,2^11]$ can be contributed to a large number of L2 cache misses. \Cref{fig:benchmark:dist-repo-cpu} plots the normalized number of instructions, which in the RAM should be directly correlated to the running time. The normalized instructions converges very quickly towards a constant, meaning that the number of retired instructions are as expected wrt. the theoretical running time. \Cref{fig:benchmark:dist-repo-cpu} also plots the L2 cache miss percentage, and it can be seen that the percentage of L2 cache misses is high when the running time of the algorithm increases more than expected. Therefore the increase in wall time can be explained by the hierarchical memory layout of a modern computer.

\begin{figure}[h!]
  \centering
  \includegraphics[width=10cm]{combined/dist_runningtime}
  \caption{Plot of the normalized running time for building the DIST repository.}
  \label{fig:benchmark:dist-repo-time}
\end{figure}

\begin{figure}[h!]
  \centering
  \includegraphics[width=10cm]{combined/dist_runningtime_cpu}
  \caption{Normalized number of retired instructions and percentage of L2 cache misses when building the DIST repository. The instructions are solid lines and the L2 cache miss percentage are dashed.}
  \label{fig:benchmark:dist-repo-cpu}
\end{figure}

\begin{figure}[h!]
  \centering
  \includegraphics[width=10cm]{combined/grid_runningtime}
  \caption{Plot of the normalized running time for filling out the grid.}
  \label{fig:benchmark:fill-grid-time}
\end{figure}

\subsubsection{Total combined running time}
It can be seen on \cref{fig:benchmark:total-time-combined} that each of the input types converges to a constant as the length of the input grows. One thing to note is that the constant for the random data and genome data is substantially higher than for the Fibonacci input \todo{give exact reason and argue why it is okay.}. Therefore the plots suggest that the combination of all the different parts of the implementation also satisfies the theoretical running time.
\begin{figure}[h!]
  \centering
  \includegraphics[width=10cm]{combined/total_runningtime}
  \caption{Plot of the normalized total running time of the combined algorithm.}
  \label{fig:benchmark:total-time-combined}
\end{figure}

\begin{figure}[h!]
  \centering
  \includegraphics[width=10cm]{combined/fib_area_plot}
  \caption{Plot of the relative running time of the different parts of the algorithm on Fibonacci input.}
  \label{fig:benchmark:relative-runningtime-fib}
\end{figure}

\begin{figure}[h!]
  \centering
  \includegraphics[width=10cm]{combined/genome_area_plot}
  \caption{Plot of the relative running time of the different parts of the algorithm on DNA sequences as input.}
  \label{fig:benchmark:relative-runningtime-genome}
\end{figure}

\begin{figure}[h!]
  \centering
  \includegraphics[width=10cm]{combined/random_area_plot}
  \caption{Plot of the relative running time of the different parts of the algorithm on random strings as input.}
  \label{fig:benchmark:relative-runningtime-random}
\end{figure}

\subsubsection{Comparison with simple algorithms}
\todo{Plots of performance compared to simple algorithm / implementation.}
\begin{figure}[h!]
  \centering
  \includegraphics[width=10cm]{combined/simple_vs_lcs}
  \caption{Plot of the running time improvement of the LCS Blow-up algorithm compared to the Simple algorithm.}
  \label{fig:benchmark:simple-vs-lcsblowup}
\end{figure}

Fibonacci input is compressed such that $n = O(\log{N})$. This means that for Fibonacci input the total running time of the algorithm becomes $O\left( N\log{N} \sqrt{\log{\frac{N}{\log{N}}}} \right) = O(N\log{N}\sqrt{\log{N}})$. Therefore running time of the LCS Blow-up method should improve by a factor $O\left( \frac{N}{\log^{\frac{3}{2}}{N}} \right)$ \todo{Does this correspond to the slope?}.

It can be seen that the LCS Blow up method is only faster for Fibonacci strings longer than $\approx 20000$ symbols.

\todo{Mention that simple algorithm can achieve a factor 4-16 speedup by using Four Russian approach.}

\subsubsection{Notes to self}
From the benchmark section of the string compression chapter (\cref{sec:compression:benchmarks}), it was found that almost all "real-world" strings, was only compressible by a constant factor (up to the trivial repetition of the string due to the fixed alphabet size \todo{experiment / comment on this?} \todo{Argue that this effect is negligible in practice (empirically determined on DNA input), and that the $\sqrt{\log{\frac{N}{n}}}$ parts eats most of the advantage}). That is, for all realistic DNA related applications it has empirically been found that $n = cN$ for some $c \in \mathbb{R}$. In this case, the running times of the algorithm is
\[
  O\left( nN\sqrt{\log{\frac{N}{n}}} \right) = O\left( c \cdot N^2 \sqrt{\log{\frac{N}{cN}}} \right) = O(N^2),
\]
which is also the running time of the simple algorithm. Therefore the practical success of the implemented algorithm becomes a battle of constants against the simple algorithm, which it is very likely going to lose. One approach to overcome this problem may be to improve the way the strings are compress, and try to make different inputs share the same SLP, a little more elaboration on this idea is done in the section on further work in \cref{chapter:conclusion}.


\chapter{Conclusion}
\label{chapter:conclusion}
Summarize findings.

\section{Further work}
\label{sec:conclusion:further-work}
During the project, a number of ideas for potentially improving (mostly of practical concerns) the approach of computing edit distance based on SLPs have been considered. These ideas are shortly described and motivated in the following list:

\begin{itemize}
  \item Parallelization of the algorithm: The two SLP representations can easily be computed in parallel. Inside the construction of the SLPs the parallelization is not obvious.
  For the expensive parts of the algorithm (constructing the DISTs and filling the dynamical programming grid), parallelization is easy. For construction of the dist, parallelization can be applied to every invocation of \texttt{build}, which happens for every pair of selected SLP productions and every recursive call.
  For filling out the grid, the algorithm can be parallelized in the same way as the simple algorithm, by choosing a suitable order to fill out the blocks.

  \item The algorithm for min-multiplying two unit-Monge matrices runs in time $O(n \log{n})$, however no lower bound other than $\Omega(n)$ is known. It could be interesting to close this gap. If is is possible to make a $\Theta(n)$ algorithm, the time for building the DIST tables can be reduced to $O(n^2 x)$, and by picking $x = \frac{N}{n}$, the total running time of the algorithm becomes $O(\underbrace{n^2 x}_{\DIST} + \underbrace{\frac{N^2}{x}}_{\text{Grid}}) = O(nN)$, which is an asymptotic improvement.

  \item Consider if the compressed strings could be reused, such that a lot of compressed strings can be entangled into one SLP. This would make it possible for a lot of strings to share DIST tables, and also possible do a better compression of highly similar strings. This could be of interest when computing the edit distance between all pairs of a large number of sequences, which is useful when approximating a phylogeny from sequence data.

  \item As mentioned in \cref{ch:compressing-strings}, the space consumption of the SLP has a very high constant when measured in terms of the number of productions in the SLP. As mentioned in the report, this is directly will not make the approach applicable to strings not compressing well, however a succinct representation of the SLP for given strings may be of interest in other problems working on SLPs.
\end{itemize}

%%%%%%%%%%%%%%%%%%%%%%%%%%%%%%%%%%%%%%%%%%%%%%%%%%%%%%%%%%%%%%%%%%%%%%%

\addcontentsline{toc}{chapter}{Bibliography}
\bibliographystyle{plain} 
\bibliography{refs}
%\addcontentsline{toc}{chapter}{Secondary Bibliography}
%\bibliographystyleB{plain} 
%\bibliographyB{refs} % remove this if you don't need secondary literature

\end{document}

